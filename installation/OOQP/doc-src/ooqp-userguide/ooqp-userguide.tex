\documentclass[twoside]{article}
\def\thefootnote{\fnsymbol{footnote}}
%\newcommand{\supp}{\bar{b}}
\newcommand{\supp}{B}
\newcommand{\suppset}{\bar{\mathcal{B}}}
%\newcommand{\slow}{\bar{\ell}}
\newcommand{\slow}{L}
\newcommand{\slowset}{\bar{\mathcal{L}}}
\newcommand{\xupp}{b}
\newcommand{\xuppset}{\mathcal{B}}
\newcommand{\xlow}{\ell}
\newcommand{\xlowset}{\mathcal{L}}
\newcommand{\dualt}{\lambda}
\newcommand{\Dualt}{\Lambda}
\newcommand{\dualu}{\pi}
\newcommand{\Dualu}{\Pi}
\newcommand{\dualv}{\gamma}
\newcommand{\Dualv}{\Gamma}
\newcommand{\dualw}{\phi}
\newcommand{\Dualw}{\Phi}

\newcommand{\rdualt}{r_{\dualt}}
\newcommand{\rdualu}{r_{\dualu}}
\newcommand{\rdualv}{r_{\dualv}}
\newcommand{\rdualw}{r_{\dualw}}

\newcommand{\Qhat}{\widehat{Q}}
\newcommand{\rhat}{\hat{r}}

\def\half  {{\textstyle{1\over 2}}}
\def\T{^T\!}
\def\minim{\mathop{\hbox{\rm minimize}}}
\def\subject{\hbox{\rm subject to}}
\def\inv{^{-1}}
\def\minim{\mathop{\hbox{\rm minimize}}}
\def\subject{\hbox{\rm subject to}}
\def\norm#1{\|#1\|}
%\input{macros}

\usepackage{subeqn}
%\usepackage{psfig}
\usepackage{extra}

\def\OOQP{OOQP}
\markboth{OOQP User Guide}{OOQP User Guide}
\author{ E. Michael GERTZ and Stephen J. WRIGHT}

\begin{document}
\pagenumbering{roman}
\setcounter{page}{1}
\thispagestyle{empty}
\begin{center}
\vspace*{-1in}
Argonne National Laboratory \\
9700 South Cass Avenue\\
Argonne, IL 60439

\vspace{.2in}
\rule{1.5in}{.01in}\\ [1ex]
ANL/MCS-TM-252 \\
\rule{1.5in}{.01in}


\vspace{1in}
{\large\bf OOQP User Guide\footnote{This 
work was supported by the Mathematical,
Information, and Computational Sciences Division subprogram of the
Office of Advanced Scientific Computing Research, U.S. Department of
Energy, under Contract W-31-109-Eng-38; and by National Science
Foundation Grants CDA-9726385 and ACI-0082065.}
}

\vspace{.2in}
by \\ [3ex]

{\large\it E. Michael Gertz\footnote{Also
Electrical and Computer Engineering Department, Northwestern University, Evanston, IL 60208; {\tt gertz@ece.nwu.edu}}
and Stephen Wright\footnote{Computer Sciences Department,
University of Wisconsin-Madison, 1210 W. Dayton Street,
Madison, WI 53706; {\tt swright@cs.wisc.edu}}}\\

\vspace{.2in}

%%%%%%%%%%%%%%%%%%%%%

\vspace{1in}
Mathematics and Computer Science Division

\bigskip

Technical Memorandum No.\ 252

\vspace{.5in}
October 2001 \\
Updated  May 2004
\end{center}

\vfill
\noindent
%%%%%%%%%%%%%%%%%%%%%


\newpage
\noindent
Argonne National Laboratory, with facilities in the states of Illinois
and Idaho, is owned by the United States Government and operated by The
University of Chicago under the provisions of a contract with the
Department of Energy.

\vspace{2in}

\begin{center}
{\bf DISCLAIMER}
\end{center}

\noindent
This
report was prepared as an account of work sponsored by an agency of the United States Government.  Neither the United States
Government nor any agency thereof, nor The University of Chicago, nor any of
their employees or officers, makes any warranty, express or implied, or assumes any legal liability or
responsibility for the accuracy, completeness, or usefulness of any
information, apparatus, product, or process disclosed, or represents that its use
would not infringe privately-owned rights.
Reference herein to any specific commercial product, process, or service by trade name,
trademark, manufacturer, or otherwise, does not necessarily constitute
or imply its endorsement, recommendation, or favoring by the United
States Government or any agency thereof.  The views and opinions of document
authors expressed herein do not necessarily state or reflect those of the
United States Government or any agency thereof, Argonne National
Laboratory, or The University of Chicago.
\newpage
  %\markboth{Users Guide for {\textsf SnadiOpt}}{}
  %\pagestyle{plain}
  \tableofcontents
\newpage
\pagenumbering{arabic}
\setcounter{page}{1}
\begin{center}
{\large\bf OOQP User Guide} \\ [2ex]
by \\ [2ex]
E. Michael Gertz and Stephen J. Wright
\end{center}
\vspace{.2in}
\addcontentsline{toc}{section}{Abstract}
\begin{abstract}
OOQP is an object-oriented software package for solving convex
quad\-ratic programming problems (QP).  We describe the design of
OOQP, and document how to use OOQP in its default configuration. We
further describe OOQP as a development framework, and outline how to
develop custom solvers that solve QPs with exploitable structure or use
specialized linear algebra.
\end{abstract}

\section{Introduction}

OOQP is a package for solving convex quadratic programming problems
(QPs). These are  optimization problems in which the objective function
is a convex quadratic function and the constraints are linear
functions of a vector of real variables. They have the following general
form:
\beq 
\label{qpintro}
\label{qp}
\min_x \, \half x^TQx + c^Tx \;\; \makebox{\rm s.t.} \;\;  
Ax=b, \; Cx \ge d,
\eeq
%
where $Q$ is a symmetric positive semidefinite $n \times n$ matrix,; $x
\in \R^n$ is a vector of variables; $A$ and $C$ are matrices of
dimensions $m_a \times n$ and $m_c \times n$, respectively; and $c$,
$b$, and $d$ are vectors of appropriate dimensions.

Many QPs that arise in practice are highly structured. That is, the
matrices that define them have properties that can be exploited in
designing efficient solution techniques. For example, they may be
general sparse matrices; diagonal, banded, or block-banded matrices;
or low-rank matrices. A simple and common instance of structure occurs
in applications in which the inequality constraints include simple
upper or lower bounds on some components of $x$; the rows of $C$
defining these bounds each contain a single nonzero element. A more
extreme example of exploitable structure occurs in the QPs that arise
in support vector machines. In one formulation of this problem, $Q$ is
dense but is a low-rank perturbation of a positive diagonal matrix.

In addition to the wide variations in problem structure, there is wide
divergence in the ways in which the problem data and variables for a
QP can be stored on a computer. Part of this variation may be due to
the structure of the particular QP: it makes sense to store the
problem data and variables in a way that is natural to the application
context in which the QP arises, rather than shoehorning it into a form
that is convenient for the QP software. Variations in storage schemes
arise also because of different storage conventions for sparse
matrices; because of the ways that matrices and vectors are
represented on different parallel platforms; and because large data
sets may necessitate specialized out-of-core storage schemes.

Algorithms for QP, as in many other areas of optimization, depend
critically on linear algebra operations of various types: matrix
factorizations, updates, vector inner products and ``saxpy''
operations.  Sophisticated software packages may be used to implement
the required linear algebra operation in a manner that is appropriate
both to the problem structure and to the underlying hardware platform.

One might expect this wide variation in structure and representation of
QPs to give rise to a plethora of algorithms, each appropriate to a
specific situation.  Such is not the case.  Algorithms such as
gradient projection, active set, and interior point all appear to
function well in a wide variety of circumstances.  Interior-point
methods in particular appear to be competitive in terms of efficiency
on almost all problem types, provided they are coded in a way that
exploits the problem structure.

In OOQP, {\em object-oriented programming} techniques are used to
implement a primal-dual interior-point algorithm in terms of abstract
operations on abstract objects.  Then, at a lower level of the code,
the abstract operations are specialized to particular problem
formulations and data representations. By reusing the top-level code
across the whole space of applications, while exploiting structure and
hardware capabilities at the lower level to produce tuned, concrete
implementations of the abstract operations, users can produce
efficient, specialized QP solvers with a minimum of effort.

This distribution of OOQP contains code to fully implement a solver
for a number of standard OOQP formulations, including a version of the
formulation \eqnok{qp} that assumes $Q$, $A$, and $C$ to be general
sparse matrices. The code in the distribution also provides a
framework and examples for users who wish to implement solvers that
are tailored to specific structured QPs and specific computational
environments.

\subsection{Different Views of OOQP}

In this section, we describe different ways in which OOQP may be used.

\paragraph{Shrink-Wrapped Solution.} 

The OOQP distribution can be used as an off-the-shelf, shrink-wrapped
solver for QPs of certain types. Users can simply install it and
execute it on their own problem data, without paying any attention to
the structure of the code or the algorithms behind it. In particular,
there is an implementation for solving general QPs (of the form
\eqnok{qpgen} given in Section~\ref{using-qpgen}) in which the data
matrices are sparse without any assumed structure. (The linear algebra
calculations in the distributed version are performed with the codes
MA27~\cite{duff82ma27}, but we have also implemented versions that use
MA57~\cite{hsl2000}, Oblio~\cite{DobP00}, and SuperLU~\cite{DemGL99}.)
The distribution also contains an implementation for computing a
support vector machine to solve a classification problem; an
implementation for solving the Huber regression problem; and an
implementation for solving a quadratic program with simple bounds on a
distributed platform, using PETSc~\cite{petsc-manual}. These implementations
each may be called via a command-line executable, using ascii input
files for defining the data in a manner appropriate to the problem.
Some of the implementations can also be called via the optimization
modeling language AMPL or via MATLAB.

See the \verb-README- file in the distribution for further details on
the specialized implementations included in the distribution.

\paragraph{Embeddable Solver.}

Some users may wish to embed OOQP code into their own applications,
calling the QP solver as a subroutine. This mode of use is familiar to
users of traditional optimization software packages and numerical
software libraries such as NAG or IMSL. The OOQP distribution supplies
C and C++ interfaces that allow the users to fill out the data arrays
for the formulation \eqnok{qpgen} themselves, then call the OOQP
solver as a subroutine.

\paragraph{Development Framework.}

Some users may wish to take advantage of the development framework
provided by OOQP to develop QP solvers that exploit the structure of
specific problems. OOQP is an extensible C++ framework; and by
defining their own specialized versions of the storage schemes and the
abstract operations used by the interior-point algorithm, users may
customize the package to work efficiently on their own applications.

Users may also modify one of the default implementations in the
distribution by replacing the matrix and vector representations and
the implementations of the abstract operations by their own
variants. For example, a user may wish to replace the code for
factoring symmetric indefinite matrices (a key operation in the
interior-point algorithms) with some alternative sparse factorization
code. Such replacements can be performed with relative ease by using
the default implementation as an exemplar.

\paragraph{Research Tool.}

Researchers in interior point-methods for convex quadratic programming
problems may wish to modify the algorithms and heuristics used in
OOQP. They can do so by modifying the top-level code, which is quite
short and easy to understand. Because of the abstraction and layering
design features of OOQP, they will then be able to see the effect of
their modifications on the whole gamut of QP problem structures
supported by the code.

\subsection{Obtaining OOQP}

The OOQP Web page \verb-www.cs.wisc.edu/~swright/ooqp/- has
instructions on downloading the distribution.
OOQP is also distributed by the Optimization Technology Center
(OTC). See the page {\tt www.ece.nwu.edu/OTC/software/} for
information on obtaining OOQP and other OTC software.

Unpacking the distribution will create a single directory called
\texttt{OOQP-X.XX}, where \texttt{X.XX} is the revision number. For
simplicity, we will refer to this directory simply as \texttt{OOQP}
throughout this document. The \texttt{OOQP} directory contains
numerous files and subdirectories, which are discussed in detail in
this manual. Whenever we refer to a particular directory in the text,
we mean it to be taken as a subdirectory of {\tt OOQP}. For example,
when we discuss the directory {\tt src/QpGen}, we mean {\tt
OOQP/src/QpGen}.

% \subsection{Problem Formulations}
% \label{sec.problem-formulations}
% 
% {\em Introduce a couple of examples of structured QPs? (Not sure if we
% need this here - we've already made the point about structured
% problems in Section~\ref{sec.whatis}.)}

\subsection{How to Read This Manual}

This manual gives an overview of OOQP---its structure,
the algorithm on which it is based, the ways in which the solvers can
be invoked, and its utility as a development framework.

Section~\ref{using-qpgen} is intended for those who wish to use the
solver for general sparse quadratic programs (formulation
\eqnok{qpgen}) that is provided with the OOQP distribution. It shows
how to define the problem and invoke the solver in various contexts.
Section~\ref{ooqp-develop-overview} gives an overview of the OOQP
development framework, explaining the basics of the layered design and
details of the directory structure and makefile-based build process.
Section~\ref{sec.qp-solver} provides additional details on the top
layer of OOQP---the QP solver layer---for the benefit of those who
wish to experiment with variations on the two primal-dual
interior-point algorithms supplied with the OOQP distribution.
Section~\ref{sec.new-qp-formulation} describes the operations that
must be defined and implemented in order to create a solver for a new
problem formulation. Section~\ref{sec.using-linear-algebra} is a
practical tutorial on OOQP's linear algebra layer. It describes the
classes for vectors and sparse and dense matrices for the benefit of
users who wish to use these classes in creating solvers for their own
problem formulations. Finally, Section~\ref{sec.specializing-linalg}
is intended for advanced users who wish to specialize the linear
algebra operations of OOQP by adding new linear solvers or using
different matrix and vector representations.
% The level of specialization relates
% not to how the QP is formulated, but rather to how data objects are
% stored and linear algebra operations performed.

Users who simply wish to use OOQP as a shrink-wrapped solver for 
quadratic programs formulated as general sparse problems \eqnok{qpgen} 
need read only Section~\ref{using-qpgen}.  Those interested in 
learning a little more about the design of OOQP should read 
Sections~\ref{sec.ooqp-develop-layers} and~\ref{sec.pdip.algorithms}, 
while those who wish to understand the design and motivation more 
fully should read Sections~\ref{sec.ooqp-develop-layers}, 
\ref{sec.qp-solver}, \ref{sec.new-qp-formulation}, and 
\ref{sec.using-linear-algebra}, in that order.  Users who wish to 
implement a solver for their own QP formulation should read 
Sections~\ref{ooqp-develop-overview}, \ref{sec.pdip.algorithms}, 
\ref{sec.new-qp-formulation}, and \ref{sec.using-linear-algebra} and 
then review Section~\ref{sec.new-qp-formulation} with code in hand.  
Those who wish to install new linear solvers should read 
Sections~\ref{using-qpgen}, \ref{ooqp-develop-overview}, 
\ref{sec.using-linear-algebra}, and then focus on 
Section~\ref{sec.specializing-linalg}.

\subsection{Other Resources}
\label{sec.other-resources}

OOQP is distributed with additional documentation. In the top-level
OOQP directory, the file README describes the contents of the
distribution. This file includes the location of an html page that
serves as an index of available documentation and may be viewed
through a browser such as Netscape. This documentation includes the
following items.
\begin{description}
  
\item[Online Reference Manual.]  We have extensively documented the
  source code , using the tool {\tt doxygen} to create a set of html
  pages that serve as a comprehensive reference manual to the OOQP
  development framework. Details of the class hierarchy, the purposes
  of the individual data structures and methods within each class, and
  the meanings of various parameters are explained in these pages.
  
\item[A Descriptive Paper.] The archival paper \cite{GerW01} by the
authors of OOQP contains further discussion of the motivation for
OOQP, the structure of the code, and the way in which the classes are
reimplemented for various specialized applications.
  
\item[Manuals for Other Problem Formulations.] Specialized QP
  formulations such as Svm, Huber, and QpBound have their own
  documentation. The documents describe the problems solved and how
  the solvers may be invoked.

\item[OOQP Installation Guide.] This document describes how to build
and install OOQP.

\item[Distribution Documents.] These include files such as README
that describe the contents of various parts of the distribution.

\end{description}

We also supply a number of sample problems and example programs in the
\texttt{examples/} subdirectory. A README file in this subdirectory
explains its contents.

%%% Local Variables: 
%%% mode: latex
%%% TeX-master: "ooqp-userguide"
%%% End: 


\newenvironment{parameters}[2]%
        {%
         \begin{list}%
          {$\bullet$}%
          {\itemsep 2pt plus 1pt minus 1pt
                \topsep 2pt plus 1pt minus 1pt 
                \settowidth{\labelwidth}{#1}
                \settowidth{\leftmargin}{#1}
%               \addtolength{\leftmargin}{\labelsep}
                \addtolength{\leftmargin}{\labelsep}}
           #2}%
        {\end{list}}
\def\parm#1{\item[\tt#1\hfill]}

\section{Using the Default QP Formulation}
\label{using-qpgen}

The ``general'' quadratic programming formulation recognized by OOQP
is as follows:
% \begin{equation}
% \label{qpgen}
% \begin{array}{crcrclc}
%   \minim   & \multicolumn{3}{r}{\half x\T Q x + c\T x} \\
%   \subject &&             & A x  & = & b \\
%            & d & \leq & C x  & \leq & f \\
%            & l & \leq & x    & \leq & u
% \end{array}
% \end{equation}
\beqa \label{qpgen}
& \min \, \half x^TQx + c^Tx \;\; \mbox{subject to} \\
\nonumber & Ax=b, \;\;\; d \le Cx \le f, \;\;\; l \le x \le u, \eeqa
where $Q$ is an $n \times n$ positive semidefinite matrix, $A$ is an
$m_a \times n$ matrix, $C$ is an $m_c \times n$ matrix, and all other
quantities are vectors of appropriate size. Some of the elements of
$l$, $u$, $d$, and $f$ may be infinite in magnitude; that is, some
components of $Cx$ and $x$ may not have upper and lower bounds.

The subdirectory {\tt src/QpGen} in the OOQP distribution, together
with the linear algebra subdirectories, contains code for solving
problems formulated as \eqnok{qpgen}, where $Q$, $A$, and $C$ are
general sparse matrices.  In this section, we describe the different
methods that can be used to define the problem data and, accordingly,
different ways in which the solver can be invoked. We start with a
command-line interface that can be used when the problem is defined
via a text file (Section~\ref{command-line}). We then describe several
other interfaces: calling OOQP as a function from a C program
(Section~\ref{embedding-c}); calling it from a C++ program
(Section~\ref{embedding-cplusplus}); invoking OOQP as a solver from an
AMPL process (Section~\ref{sec.use-in-ampl}); and invoking OOQP as a
subroutine from a MATLAB program (Section~\ref{sec.use-in-matlab}).

\subsection{Command-Line Interface} 
\label{command-line}

When the problem is defined in quadratic MPS (``QPS'') format in an
ascii file, the method of choice for solving the problem is to use an
executable file that applies Mehrotra's predictor-corrector algorithm
\cite{Meh92a} with Gondzio's multiple corrections~\cite{Gon94d}. (The
Installation Guide that is supplied with the OOQP distribution
describes how to create this executable file, which is named {\tt
  qpgen-sparse-gondzio.exe}.)  We also provide
\texttt{qpgen-sparse-mehrotra.exe}, an implementation of Mehrotra's
algorithm that does not use Gondzio's corrections.  These executables
take their inputs from a text file in QPS format that describes the
problem.

% This program accepts a single command-line argument; a typical
% invocation has the form
% \begin{verbatim}
% qpgen-sparse-gondzio.exe problem.qps
% \end{verbatim}
% where \texttt{problem.qps} is the name of a file that contains the
% specification of \eqnok{qpgen} in QPS format.

The QPS format proposed by Maros and M\'esz\'aros~\cite{MarM99} is a
modification of the standard and widely used MPS format for linear
programming.  The format is somewhat awkward and limited in the
precision to which it can specify numerical data. We support it,
however, because it is used by a number of QP solvers and is well
known to users of optimization software.

A description of the MPS format, extracted from Murtagh~\cite{Mur81},
can be found at the NEOS Guide at
\begin{verbatim}
www.mcs.anl.gov/otc/Guide/
\end{verbatim}
(search for ``MPS'').  The QPS format extends MPS by introducing a new
section of the input file named \texttt{QUADOBJ} (alternatively named
\texttt{QMATRIX}), which defines the matrix $Q$ of the quadratic
objective specified in the formulation~(\ref{qpgen}). This section, if
present, must appear after all other sections in the input file. The
format of this section is the same as the format of the
\texttt{COLUMNS} section except that only the lower triangle of $Q$ is
stored. As in the \texttt{COLUMNS} section, the nonzeros are specified
in column major order.

We have relaxed the MPS definition so that strict limitations on field
widths on each line are replaced by tokenization, in which fields are
assumed to be separated by spaces.  (Note that this parsing may
introduce incompatibilities with files that are valid under the strict
MPS definition, in which spaces may occur within a single numerical
field between a minus sign and the digits it operates on.) Name
records may now be up to 16 characters in length, and there is no
restriction on the size of numerical fields, except those imposed by
the maximum length of a line. The maximum line length is 150
characters.

A second deviation from MPS standard format is that an objective sense
indicator may be introduced at the start of the file, to indicate
either that the specified objetcive is to be minimized or maximized.
This field has the form
\begin{verbatim}
OBJSENSE
MIN
\end{verbatim}
when the intent is to minimize the objective, and
\begin{verbatim}
OBJSENSE
MAX
\end{verbatim}
for maximization. The default is minimization. If this field is
included, it must appear immediately after the \texttt{NAME} line.

% In an MPS file, line breaks are significant, as is the location of
% data within a line. Throughout this section, we will refer to the
% first character on a line as character one. A data line in an MPS file
% is divided into two or more fields.Each field occurs at a fixed
% location withina data line. These locations are given in
% Table~\ref{fields}.
% \begin{table}
%   \label{fields}
%    \caption{Field locations in an MPS data line}
% \begin{tabular}{|l|cccccc|}
%   \hline
%            & Field 1 & Field 2 & Field 3 & Field 4 & Field 5 & Field 6
%            \\
% \hline
% Characters &  2--3   &  5--12  & 15--22  & 25--36  & 40--47  & 50--61 \\
% \hline
% \end{tabular}
% \end{table}

\begin{figure}[hbt]
\begin{verbatim}
NAME          Example
ROWS
 N  obj
 G  r1
 L  r2
COLUMNS
    x1        r1                 2.0   r2                -1.0
    x1        obj                1.5
    x2        r1                 1.0   r2                 2.0
    x2        obj               -2.0
RHS
    rhs1      obj               -4.0
    rhs1      r1                 2.0   r2                 6.0
BOUNDS
 UP bnd1      x1                20.0
QUADOBJ
    x1        x1                 8.0
    x1        x2                 2.0
    x2        x2                10.0
ENDATA
\end{verbatim}
\caption{A sample QPS file\label{qps-example}} 
\end{figure}
Figure~\ref{qps-example} shows a sample QPS file, taken from Maros and
M\'esz\'aros~\cite{MarM99}.
This file describes the following problem:
\begin{equation}
  \label{qp-example}
  \begin{array}{lrcccl}
    \mathrm{mimimize} & 
    \multicolumn{5}{l}{4 + 1.5 x_1 -2 x_2 + \frac12(8x_1^2 + 
      4x_1 x_2 + 10 x_2^2)} \\
    \mbox{subject to} & 2 & \leq & 2 x_1 +   x_2 & \leq & \infty \\
                      & -\infty  & \leq   & - x_1 + 2 x_2 & \leq & 6 \\
                      & 0 & \leq & x_1         & \leq & 20 \\
                      & 0 & \leq & x_2         & \leq & \infty. \\
  \end{array}
\end{equation}
If the file \eqnok{qps-example} is named \texttt{Example.qps} and is
stored in the subdirectory {\tt data}, and if the executable {\tt
qpgen-sparse-gondzio.exe} appears in the {\tt OOQP} directory, then
typing
\begin{verbatim}
qpgen-sparse-gondzio.exe ./data/Example.qps
\end{verbatim}
will solve the problem
and create the output file
\texttt{OOQP/data/Example.out}. 

\begin{figure}[hbt]
\begin{verbatim}
Solution for 'Example '

Rows: 3,  Columns: 2

PRIMAL VARIABLES

  Name  Value           Lower Bound  Upper Bound  Multiplier

0 x1    7.62500000e-01  0.00000e+00  2.00000e+01  6.37776644e-15
1 x2    4.75000000e-01  0.00000e+00               2.83645544e-12


CONSTRAINTS

Inequality Constraints: 2

  Name  Value           Lower Bound  Upper Bound  Multiplier

0 r1    2.00000000e+00  2.00000e+00               4.27500000e+00
1 r2    1.87500000e-01               6.00000e+00  -8.81876986e-16

Objective value: 8.37188
\end{verbatim}
\caption{Sample output from \texttt{qpgen-sparse-gondzio.exe}\label{sample-output}}
\end{figure}
Figure~\ref{sample-output} shows the contents of \texttt{Example.out}.
The {\tt PRIMAL VARIABLES} are the components of the vector $x$ in the
formulation \eqnok{qpgen}. The output shows that the optimal values
are $x_1 = 0.7625$ and $x_2=0.475$.  The bounds on each component of
$x$, if any were specified, are also displayed. If neither bound is
active, the reported value of the Lagrange multiplier in the final
column should be close to zero. Otherwise, it may take a positive
value when the lower bound is active or a negative value when the
upper bound is active.

The \texttt{CONSTRAINTS} section shows the values of the vectors $Ax$
and $Cx$ at the computed solution $x$, compares these values with
their upper and lower bounds in the case of $Cx$, and displays
Lagrange multiplier information in the final column, in a similar way
to the {\tt PRIMAL VARIABLES} section.

Note that the problem described in \eqnok{qps-example} contains no
equality constraints (that is, $A$ is null), so there is no
\texttt{Equality Constraints} subsection in the \texttt{CONSTRAINTS}
section of this particular output file.

A number of command-line options are available in calling
\texttt{qpgen-sparse-gondzio.exe}. The current list of options can be
seen by typing 
\begin{verbatim}
qpgen-sparse-gondzio.exe --help
\end{verbatim}
Current options are as follows:
\begin{description}
\item[{\tt --print-level num}:] (\texttt{num} is a positive
  integer)  Larger values of \texttt{num} produce more output to the
  screen.

\item[{\tt --version}:]  shows the current version number and release date.

\item[{\tt --quiet}:] suppresses output  to the screen.

\item[{\tt --verbose}:] produces maximal output to the screen.
  
\item[{\tt --scale}:] scales the variables so that the diagonals of the
  Hessian matrix remain in a reasonable range.
\end{description}

The same options are available for \texttt{qpgen-sparse-mehrotra.exe}.

\subsection{Calling from a C Program}
\label{embedding-c}

OOQP supplies an interface to the default solver for \eqnok{qpgen} that
may be called from a C program.  This operation is performed by the
function \texttt{qpsolvesp}, which has the following prototype.
\begin{verbatim}
void 
qpsolvesp(          double    c[],   int      nx,
    int   irowQ[],  int    nnzQ,     int  jcolQ[],  double dQ[],
    double xlow[],  char  ixlow[], 
    double xupp[],  char  ixupp[],
    int   irowA[],  int     nnzA,    int  jcolA[],  double dA[],
    double   bA[],  int       my,
    int   irowC[],  int     nnzC,    int  jcolC[],  double dC[],
    double clow[],  int       mz,    char iclow[],
    double cupp[],  char   icupp[],
    double    x[],  double gamma[],  double phi[],
    double    y[],
    double    z[],  double lambda[],  double pi[],
    double   *objectiveValue,
    int print_level, int * ierr );
\end{verbatim}
This function uses an old-fashioned calling convention in which each
argument is a native type (for example, an {\tt int} or an array of
{\tt double}). While calling such a function can be tedious because of
the sheer number of arguments, it is straightforward in that the
relationship of each argument to the formulation \eqnok{qpgen} is
fairly easy to understand.

Sparse matrices are represented by three data structures---two integer
vectors and one vector of doubles, all of the same length. For the
(general) matrix $A$, these data structures are \texttt{irowA},
\texttt{jcolA} and \texttt{dA}. The total number of nonzeros in the
sparse matrix \texttt{A} is \texttt{nnzA}. The $k$ nonzero
element of \texttt{A} occurs at row \texttt{irowA[k]} and column
\texttt{jcolA[k]} and has value \texttt{dA[k]}. Rows and columns are
numbered starting at zero.

For the symmetric matrix $Q$, only the elements of the lower triangle
of the matrix are specified in \texttt{irowQ}, \texttt{jcolQ}, and
\texttt{dQ}.

The elements of each matrix must be sorted into row-major order before
\texttt{qpsolvesp} is called.  While this requirement places an
additional burden on the user, it reduces the memory requirements of
the \texttt{qpsolvesp} procedure significantly. OOQP provides a
routine \texttt{doubleLexSort} that the user may call to sort the
matrix elements in the correct order. To sort the elements of the
matrix $A$, this routine can be invoked as follows:
\begin{verbatim}
doubleLexSort( irowA, nnzA, jcolA, dA )
\end{verbatim}

We now show the correspondence between the input variables to
\texttt{qpsolvesp} (which are not changed by the routine) and the
formulation \eqnok{qpgen}.
\begin{parameters}{\texttt{lambda}}
  \parm{c} is the linear term in the objective function, a vector of
length \texttt{nx}.
  
  \parm{nx} is the number of primal variables, that is, the length of
the vector $x$ in \eqnok{qpgen}. It is the length of the input vectors
\texttt{c}, \texttt{xlow}, \texttt{ixlow}, \texttt{xupp},
\texttt{ixupp}, \texttt{x}, \texttt{gamma}, and \texttt{phi}.

  \parm{irowQ, jcolQ, dQ} hold the \texttt{nnzQ} lower triangular
elements of the quadratic term of the objective function.

  \parm{xlow, ixlow} are the lower bounds on \texttt{x}. These contain
the information in the lower bounding vector $l$ in \eqnok{qpgen}. If
there is a bound on element $k$ of $x$ (that is, $l_k > -\infty$),
then \texttt{xlow[k]} should be set to the value of $l_k$ and
\texttt{ixlow[k]} should be set to one. Otherwise, element $k$ of both
arrays should be set to zero.

  \parm{xupp, ixupp} are the upper bounds on \texttt{x}, that is, the
information in the vector $u$ in \eqnok{qpgen}. These should be
defined in a similar fashion to \texttt{xlow} and \texttt{ixlow}.

  \parm{irowA, jcolA, dA} are the \texttt{nnzA} nonzero elements of
the matrix $A$ of linear equality constraints.
  
  \parm{irowC, jcolC, dC} are the \texttt{nnzC} nonzero elements of
the matrix $C$ of linear inequality constraints.

  \parm{bA} contains the right-hand-side vector $b$ for the equality
constraints in \eqnok{qpgen}. 

  \parm{my} defines the length of the vector $bA$.

%  \parm{y} defines the vector of Lagrange multipliers for the equality
%  constraints $Ax=b$ in \eqnok{qpgen}. It has length \texttt{my}.

  \parm{clow, iclow} are the lower bounds of the inequality
  constraints.
  
  \parm{cupp, icupp} are the upper bounds of the inequality
  constraints.

  \parm{print\_level} controls the amount of output the solver prints
  to the terminal. Larger values of \texttt{print\_level} cause more
  information to be printed. The following values of
  \texttt{print\_level} are recognized:
  \begin{description}
    \item{$0$} operate silently.
    \item{$\geq 10$} print information about each interior point
      iteration.
    \item{$\geq 100$} print information from the linear solvers.
    \end{description}
\end{parameters}

The remaining parameters are output parameters that hold the solution
to the QP. The variables \texttt{objectiveValue} and \texttt{x} hold
the values of interest to most users, which are the minimal value and
solution vector $x$ in \eqnok{qpgen}. The parameter \texttt{ierr}
indicates whether the solver was successful. The solver will return a
nonzero value in \texttt{ierr} if it was unable to solve the problem.
Negative values indicate that some error, such as an out of memory
error, was encountered. For a description of the termination criteria
of OOQP, and the positive values that might be returned in
\texttt{ierr}, see Section~\ref{sec.status}.


The remaining output variables are vectors of Lagrange
multipliers; the array \texttt{y} contains the Lagrange multipliers
for the equality constraints $Ax=b$, while \texttt{lambda} and
\texttt{pi} contain multipliers for the inequality constraints $Cx \ge
d$ and $Cx \le f$, respectively. The output variable \texttt{z} should
satisfy
$$
z = \lambda - \pi.
$$
The multipliers for the lower and upper bounds $x \ge l$ and $x \le
u$, are contained in \texttt{gamma} and \texttt{phi}, respectively.
Among other requirements (see our discussion of optimality conditions
in the next section), these vectors should satisfy the following
relationship on output:
\[
c + Q x - A\T y - C\T z - \gamma + \phi = 0.
\]

Because it is somewhat cumbersome to allocate storage for all the
parameters of \texttt{qpsolvesp} individually, OOQP provides the
following routine to perform all necessary allocations:
\begin{verbatim}
void 
newQpGenSparse( double ** c,      int nx,
    int    ** irowQ,  int nnzQ,  int  ** jcolQ,  double ** dQ,
    double ** xlow,              char ** ixlow,
    double ** xupp,              char ** ixupp,
    int    ** irowA,  int nnzA,  int  ** jcolA,  double ** dA,
    double ** b,      int my,
    int    ** irowC,  int nnzC,  int  ** jcolC,  double ** dC,
    double ** clow,   int mz,    char ** iclow,
    double ** cupp,              char ** icupp,
    int    *  ierr );
\end{verbatim}

The following routine frees all this allocated storage:
\begin{verbatim}
void 
freeQpGenSparse( double ** c,      
    int    ** irowQ,  int  ** jcolQ,  double ** dQ,
    double ** xlow,   char ** ixlow,
    double ** xupp,   char ** ixupp,
    int    ** irowA,  int  ** jcolA,  double ** dA,
    double ** b,
    int    ** irowC,  int  ** jcolC,  double ** dC,
    double ** clow,   char ** iclow,
    double ** cupp,   char ** icupp );
\end{verbatim}
If \texttt{newQpGenSparse} succeeds, it returns \texttt{ierr} with a
value of zero. Otherwise, it sets \texttt{ierr} to a nonzero value
and frees any memory that it may have allocated to that point. We
emphasize that users are not required to use these two routines; users
can allocate arrays as they choose.

The distribution also contains a variant of \texttt{qpsolvesp} that
accepts sparse matrices stored in the slightly more compact
Harwell-Boeing sparse format (see Duff, Erisman, and
Reid~\cite{DufER86}), rather than the default sparse format described
above. In the Harwell-Boeing format, the nonzeros are stored in
row-major form, with \texttt{jcolA[l]} and \texttt{dA[l]} containing
the column index and value of the $l$ nonzero element, respectively.
The integer vector \texttt{krowA[k]} indicates the index in
\texttt{jcolA} and \texttt{dA} at which the first nonzero element for
row \texttt{k} is stored; its final element \texttt{krowA[my+1]}
points to the index in \texttt{jcolA} and \texttt{dA} immediately
after the last nonzero entry. See \cite{DufER86} and
Section~\ref{sec.using-sparse} below for further details.  The
Harwell-Boeing version of \texttt{qpsolvesp} has the following
prototype.
\begin{verbatim}
void 
qpsolvehb( double    c[],  int  nx,
    int   krowQ[],  int  jcolQ[],  double dQ[],
    double xlow[],  char ixlow[], 
    double xupp[],  char ixupp[],
    int   krowA[],  int  my,       int  jcolA[],  double dA[],
    double   bA[],
    int   krowC[],  int  mz,       int  jcolC[],  double dC[],
    double clow[],  char iclow[],
    double cupp[],  char icupp[],
    double    x[],  double gamma[],     double phi[],
    double    y[],
    double    z[],  double lambda[],  double pi[],
    double   *objectiveValue,
    int print_level, int * ierr );
\end{verbatim}
The meaning of the parameters other than those that store the sparse
matrices is identical to the case of \texttt{qpsolvesp}.



The prototypes of the preceding routines are located in the header
file \texttt{cQpGenSparse.h}.  Most users will need to include the
line
\begin{verbatim}
#include "cQpGenSparse.h"
\end{verbatim}
in their program. This header file is safe to include not only in a C
program but also in a C++ program. Users who need more control over
the solver than these functions provide should develop a C++ interface
to the solver.

We refer users to the Installation Guide in the distribution for
further information on building the executable using the OOQP header
files and libraries.


\subsection{Calling from a C++ Program}
\label{embedding-cplusplus}

When calling OOQP from a C++ code, the user must create several objects
and call several methods in sequence.  The process is more complicated
than simply calling a C function, but also more flexible. By varying
the classes of the objects created, one can generate customized
solvers for QPs of various types. In this section, we focus on
the default solver for the formulation \eqnok{qpgen}. The full
sequence of calls for this case is shown in Figure~\ref{calling}. In
the remainder of this section, we explain each call in this sequence
in turn.

\begin{figure}[htb]
\begin{verbatim}
QpGenSparseMa27 * qp 
    = new QpGenSparseMa27( nx, my, mz, nnzQ, nnzA, nnzC );

QpGenData      * prob 
    = (QpGenData * ) qp->makeData( /* parameters here */);
QpGenVars      * vars 
    = (QpGenVars *) qp->makeVariables( prob );
QpGenResiduals * resid 
    = (QpGenResiduals *) qp->makeResiduals( prob );

GondzioSolver  * s     = new GondzioSolver( qp, prob );

s->monitorSelf();
int status = s->solve(prob,vars, resid);
\end{verbatim}
\caption{The basic sequence for calling OOQP \label{calling}}
\end{figure}

The first method call in this sequence 
% is as follows.
% \begin{verbatim}
% QpGenSparseMa27 * qp 
%     = new QpGenSparseMa27( nx, my, mz, nnzQ, nnzA, nnzC );
% \end{verbatim}
% This call 
initializes a new problem formulation \texttt{qp} of class
\texttt{QpGenSparseMa27}, which is a subclass of
\texttt{ProblemFormulation}.  The definition of this class determines
how the problem data will be stored, how the problem variables will be
stored and manipulated, and how linear systems will be solved. Our
subclass \texttt{QpGenSparseMa27} implements the problem formulation
\eqnok{qpgen}, where the sparse matrices defining the problem are
stored in sparse (not dense) matrices and that large linear systems
that define steps of the interior-point method will be solved by using
the \texttt{MA27} package from the Harwell Subroutine Library.

% The next method call is
% \begin{verbatim}
% QpGenData * prob 
%     = (QpGenData * ) qp->makeData( /* parameters here */);
% \end{verbatim}
% Here, 
In the next method call in Figure~\ref{calling},
the \texttt{makeData} method in the object \texttt{qp} created
in the first call creates the vectors and matrices that contain the
problem data. In fact, \texttt{qp} contains different versions of the
\texttt{makeData} method, which may be distinguished by their
different parameter lists. Users whose matrix data is in row-major
Harwell-Boeing sparse format may use the following form of this call.

\pagebreak
\begin{verbatim}
QpGenData * prob 
    = (QpGenData * ) qp->makeData( c,     krowQ,  jcolQ,  dQ,
                                   xlow,  ixlow,  xupp,   ixupp,
                                   krowA, jcolA,  dA,     bA,
                                   krowC, jcolC,  dC,
                                   clow,  iclow,  cupp, icupp);
\end{verbatim}
(The meaning of the parameters is explained in
Section~\ref{embedding-c} above.)  In this method, data structure
references in \texttt{prob} are set to the actual arrays given in the
parameter list.  This choice avoids copying of the data, but it
requires that these arrays not be deleted until after deletion of the
object \texttt{prob}.

For users whose data is in sparse triple format, a special version of
\texttt{makeData} named \texttt{copyDataFromSparseTriple} may be
called as follows.
\begin{verbatim}
QpGenData * prob 
    = (QpGenData * ) qp->copyDataFromSparseTriple(
        c,      irowQ,  nnzQ,   jcolQ,  dQ,
        xlow,   ixlow,  xupp,   ixupp,
        irowA,  nnzA,   jcolA,  dA,     bA,
        irowC,  nnzC,   jcolC, 
        clow,   iclow,  cupp,   icupp );
\end{verbatim}
(The meaning of the parameters is explained in
Section~\ref{embedding-c}.) In this method, since the data
objects in the argument list are actually copied into \texttt{prob},
they may be deleted immediately after the method returns.

There distribution includes several other version of \texttt{makeData}
that will not be described here. In general, the preference is to fix
references in \texttt{prob} to point to existing arrays of data,
rather than copying the data into \texttt{prob}.

The calls to \texttt{makeVariables} and \texttt{makeResiduals} in
Figure~\ref{calling} create the objects that store the problem
variables and the residuals that measure the infeasibility of a given
point with respect to the various optimality conditions.
The object \texttt{vars} contains both primal variables for
\eqnok{qpgen} (including $x$) and dual variables (Lagrange
multipliers). These variables are named \texttt{vars->x},
\texttt{vars->y}, and so on, following the naming conventions
described in Section~\ref{embedding-c}. The data and methods in
the residuals class \texttt{resids} are typically of interest only to
optimization experts. When an approximate solution to the problem
\eqnok{qpgen} is found, all data elements in this object will have
small values, indicating that the point in question approximately
satisfies all optimality conditions.

The next step is to create the solver object for actually solving the
QP. This is performed by means of the following call.
\begin{verbatim}
GondzioSolver  * s     = new GondzioSolver( qp, prob );
\end{verbatim}
In our example, we then invoke the method \texttt{s->monitorSelf()}
to tell the solver that it should print summary information to the
screen as the solver is operating.  (If this line is omitted, the
solver will operate quietly.)  

Finally, we invoke the algorithm to solve
the problem by means of the call  {\tt s->solve(prob,vars, resid)}.
% \begin{verbatim}
% s->solve(prob,vars, resid);
% \end{verbatim}
The return value from this routine will be zero if the solver was able
to compute an approximate solution, which will be found in the object
\texttt{vars}. The solver will return a nonzero value if it was unable
to solve the problem. Negative values indicate that some error, such
as an out of memory error, was encountered. For a description of the
termination criteria of OOQP, and the positive values that might be
returned in \texttt{ierr}, see Section~\ref{sec.status}.

One must include certain header files to obtain the proper definitions
of the classes used. In general, a class definition is in a header
file with the same name as the class, appended with a ``.h''. For the
example in Figure~\ref{calling}, the following lines serve to include
all relevant header files.
\begin{verbatim}
#include "QpGenData.h"
#include "QpGenVars.h"
#include "QpGenResiduals.h"
#include "GondzioSolver.h"
#include "QpGenSparseMa27.h"
\end{verbatim}
The OOQP Installation Guide explains how to build an executable using
the OOQP header files and libraries.

\subsection{Use in AMPL}
\label{sec.use-in-ampl}

OOQP may be invoked within AMPL, a modeling language for specifying
optimization problems.  From within AMPL, one must first define the
model and input the data. If the model happens to be a QP, then an
{\tt option solver} command within the AMPL code can be used to ensure
the use of OOQP as the solver.

An AMPL model file that may be used to describe a problem of the form
(\ref{qpgen}) without equalities $Ax=b$ is as follows.
\begin{verbatim}
set I;  set J;
set QJJ within {J,J}; set CIJ within {I,J};

param objadd;  param g{J};   param Q{QJJ};
param clow{I}; param C{CIJ}; param cupp{I};
param xlow{J}; param xupp{J};

var x{j in J} >= xlow[j] <= xupp[j];

minimize total_cost: objadd + sum{j in J} g[j] * x[j] 
         + 0.5 * sum{(j,k) in QJJ} Q[j,k] * x[j] * x[k];

subject to ineq{i in I} :
        clow[i] <= sum{(i,j) in CIJ } C[i,j] * x[j] <= cupp[i] ;
\end{verbatim}
The data for the QP is normally given in a separate AMPL data file,
which for the problem (\ref{qp-example}) is as follows.
\begin{verbatim}
param  objadd   := 4 ;

param: J : g    :=  col1    1.5         col2   -2 ;

param: QJJ : Q := 
       col1    col1    8    col1    col2    2
       col2    col1    2    col2    col2    10 ;

param xlow      :=  col1    0           col2    0 ;
param xupp      :=  col1    20          col2    Infinity ;

param: I : clow :=  row1    2           row2   -Infinity ;  
param      cupp :=  row1    Infinity    row2    6 ;      
      
param: CIJ : C := 
       row1    col1    2    row1    col2    1
       row2    col1   -1    row2    col2    2 ;
\end{verbatim}
Suppose the model file was named \texttt{example.mod} and the data
file was named \texttt{example.dat}. From within the AMPL environment,
one would type the following lines to solve the problem and view the
solution.
\begin{verbatim}
model example.mod;
data  example.dat;
option solver ooqp-ampl;
solve;
display x;                                                                
\end{verbatim}
The following lines containing the optimal value of $x$ would then be
displayed.
\begin{verbatim}
x [*] :=
col1  0.7625
col2  0.475
;
\end{verbatim}
                                                     
\subsection{Use in MATLAB}
\label{sec.use-in-matlab}

OOQP may be invoked from within the MATLAB environment. Instructions
on how to obtain the necessary software may be found in the
\texttt{README-Matlab} in the {\tt OOQP} directory.

The prototype for the MATLAB function is as follows.
\begin{verbatim}
[x, gamma, phi, y, z, lambda, pi] = ...
    ooqp( c, Q, xlow, xupp, A, dA, C, clow, cupp, doPrint )
\end{verbatim}
This function will solve the general QP formulation (\ref{qpgen}),
re-expressed here in MATLAB notation.
\begin{verbatim}
minimize:     c' * x + 0.5 * x' * Q * x
subject to:   A * x = dA
              clow <= C * x <= cupp
              xlow <=     x <= xupp
\end{verbatim}
This is the exactly the default QP formulation~\eqnok{qpgen}. The
vectors and matrix objects in the argument list should be MATLAB
matrices of appropriate size.  Upper or lower bounds that are absent
should be set to \texttt{inf} or \texttt{-inf}, respectively. (It is
important to use these infinite values rather than large but finite
values.)

The final parameter in the argument list, \texttt{doPrint}, is
optional. If present, it should be set to one of the strings ``yes,''
``on,'' ``no,'' or ``off.'' If the value is ``yes'' or ``on,'' then
progress information will be printed while the algorithm solves the
problem. If \texttt{doPrint} is absent, the default value ``off'' will
be assumed.

Help is also available within MATLAB. After you have followed the
instruction in \texttt{README-Matlab} and installed the MATLAB
interface in the local directory or on the MATLAB path, help can by
obtained by typing \texttt{help ooqp} at the MATLAB prompt.

%%% Local Variables: 
%%% mode: latex
%%% TeX-master: "ooqp-userguide"
%%% End: 

\section{Overview of the OOQP Development Framework}
\label{ooqp-develop-overview}

In this section, we start by describing the layered design of OOQP,
which is the fundamental organizing principle for the classes that
make it up. We then discuss the directory structure of the OOQP
distribution, and the makefile-based build process that is used to
construct executables.

\subsection{The Three Layers of OOQP}
\label{sec.ooqp-develop-layers}

OOQP has a layered design in which each layer is built from
abstract operations defined by the layer below it. We sketch these
layers and their chief components in turn.

\paragraph{QP Solver Layer.} The top layer of OOQP contains the high-level
algorithms and heuristics for solving quadratic programming
problems. OOQP implements primal-dual interior-point algorithms, that
are motivated by the optimality (Karush-Kuhn-Tucker) conditions for a
QP. We write these conditions for the formulation \eqnok{qpintro} by
introducing Lagrange multiplier vectors $y$ and $z$ (for the equality
and inequality constraints, respectively) and a slack vector $s$ to
yield the following system:
\begin{subequations} \label{resids}
\beqa
\label{resids.1}
c + Q x -  A^T y - C^T z & = & 0, \\
\label{resids.2}
A x - b & = & 0, \\
\label{resids.3}
C x - s - d & = & 0, \\
\label{resids.4}
S Z e & = & 0, \\
\label{resids.5}
s, z & \ge & 0,
\eeqa
\end{subequations}
where $S$ and $Z$ are diagonal matrices whose diagonal elements are
the components of the vectors $s$ and $z$, respectively. A primal-dual
interior-point algorithm finds a solution to \eqnok{qpintro} by
applying Newton-like methods to the nonlinear system of equations
formed by \eqnok{resids.1}, \eqnok{resids.2}, \eqnok{resids.3}, and
\eqnok{resids.4}, constraining all iterates $(x^k,y^k,z^k,s^k)$,
$k=0,1,2,\dots$ to satisfy the bounds \eqnok{resids.5} {\em strictly}
(that is, all components of $z^k$ and $s^k$ are strictly positive for
all $k$). 

OOQP implements the primal-dual interior point algorithm of
Mehrotra~\cite{Meh92b} for linear programming, and the variant
proposed by Gondzio~\cite{Gon94d} that includes higher-order
corrections.  See Section~\ref{sec.pdip.algorithms} below, and the
text of Wright~\cite{IPPD96} for further description of these methods.

\paragraph{Problem Formulation Layer.}

Algorithms in the QP solver layer are built entirely from abstract
operations defined in the problem formulation layer. This layer
consists of several classes each of which represents an object of
interest to a primal-dual interior-point algorithm. The major classes
are as follows.
\begin{description}
  
\item[Data] Stores the data $(Q, A, C, c, b, d)$ defining the QP
\eqnok{qpintro}, in an economical format suited to the specific
structure of the data and the operations needed to perform on it.
  
\item[Variables] Contains storage for the primal and dual variables
$(x, y, z, s)$ of the problem, in a format suited to the specific
structure of the problem.  Also implements various methods associated
with the variables, including the computation of a maximum steplength,
saxpy operations, and calculation of $\mu = (s\T z)/m_{C}$.
  
\item[Residuals] Contains storage for the residuals---the vectors that
indicate infeasibility with respect to the KKT conditions---along with
methods to calculate these residuals from formulae such as
(\ref{resids.1}--\ref{resids.4}).  This class also contains methods for
performing the projection operations needed by the Gondzio approach,
calculating residual norms, and calculating the current duality gap
(see Section~\ref{sec:residualsclass} for a discussion of the duality
gap.)
  
\item[LinearSystem] Contain methods to factor the coefficient matrix
used in the Newton-like iterations of the QP solver and methods that
use the information from the factorization to solve the linear systems
for different right-hand sides. The systems that must be solved are
described in Section~\ref{sec.pdip.algorithms}.

\end{description}

To be concrete in our discussion, we have referred to the QP
formulation~(\ref{qpintro}) given in the introduction, but the problem
formulation layer provides abstract operations suitable to many
different problem formats. For instance, the quadratic program that
arises from classical support vector machine problems is
\begin{equation}
  \label{svm-formulation}
    \min  \norm{w}^2 + \rho e\T u, \ 
    \subject \ D ( V w - \beta e ) \geq  e - u, \;\; v \ge 0,
\end{equation}
where $V$ is a matrix of empirical observations, $D$ is a diagonal
matrix whose entries are $\pm 1$, $\rho$ is a positive scalar
constant, and $e$ is a constant vector of all ones. In OOQP's
implementation of the solver for this problem, we avoid expressing the
problem in the form \eqnok{qpgen} by forming the matrices $Q$ and $C$
explicitly. Rather, the problem formulation layer provides methods to
perform operations involving $Q$, $C$, and the other data objects that
define the problem. The QP solver layer implements a solver by calling
these methods, rather than operating on the data and variables
explicitly.

Since a solver for general problems of the form \eqnok{qpgen} is
useful in many circumstances, OOQP provides a solver for this
formulation, as well as for several specialized formulations such as
\eqnok{svm-formulation}. Users may readily specialize the abstract
operations in this layer and thereby create solvers that are
specialized to yet more problem formulations.
Section~\ref{sec.new-qp-formulation} gives instructions on how to
develop specialized implementations of this class.
 
\paragraph{Linear Algebra Layer.} 

Many of the linear algebra operations and data structures in OOQP are
shared by several problem types.  For instance, regardless of the
particular QP formulation, the Variable, Data, and LinearSystems
classes will need to perform saxpy, dot product, and norm calculations
on vectors of doubles.  Furthermore, most sparse problems will need to
store matrices in a Harwell-Boeing format. Reimplementing the linear
algebra operations in each of the problem-dependent classes would
result in an unacceptable explosion in code size and complexity. The
solution we implemented in OOQP is to define another layer that
provides the linear algebra functionality needed by many different
problem formulations. An added advantage is that by confining linear
algebra to its own layer, we can implement solvers for distributed
platforms with little change in the code.

The linear algebra classes are somewhat a different from
the classes in the QP solver and problem formulation layers. The two
topmost layers of OOQP consist of small, abstract interfaces with no
behavior whatsoever. We have provided concrete implementations based
on these interfaces, but even our concrete classes tend to contain
only a small number of methods. Hence, these classes are easy to
understand and easy to override.

By contrast, implementations of linear algebra classes such as
\texttt{DoubleMatrix} and \texttt{OoqpVector} must supply a relatively
large amount of behavior. This complexity appears to be inevitable.
The widely used BLAS library, which is meant to contain only the most
basic linear algebra operations, consists of forty-nine families of
functions and subroutines.
% (each family consists of versions of the
% routine for different numerical precisions.) 
As well as defining operations, the linear algebra classes also have
to handle the storage of their component elements.

Our approach to the linear algebra classes is to identify and provide
as methods the basic operations that are used repeatedly in our
implementations of the problem formulation layer. As much as possible,
we use existing packages such as BLAS~\cite{lawson79basic},
LAPACK~\cite{lapack} and
PETSc~\cite{petsc-home-page,petsc-efficient,petsc-manual} to supply
the behavior needed to implement these methods. Since our goal is
simplicity, we provide only the functionality that we use. We are not
striving for a complete linear algebra package but for a package
that may be conveniently used in the setting of interior point
optimization algorithms. For this reason, many BLAS operations are not
provided; and certain operations common in interior-point algorithms,
but rare elsewhere, are given equal status with BLAS routines.


\subsection{OOQP Directory Structure and Build Process}

The OOQP installation process will generate compiled libraries in the
directory \texttt{lib} and a directory named \texttt{include}
containing header files.  These libraries and headers may be copied
into a more permanent system-dependent location. Users who wish to
call OOQP code from within their own C or C++ programs may use any
build process they wish to compile and link against the installed
headers and libraries.

Users who wish to do more complex development with OOQP may find it
more convenient to work within the source directory \texttt{src}
and use the OOQP build system to compile their executables. OOQP has a
modular directory structure in which source and header files that
logically belong together are placed in their own subdirectory of
\texttt{src}. For example, code that implements the solver for
the formulation \eqnok{qpgen} can be found in \texttt{src/QpGen},
while code that defines classes for dense matrices and dense linear
equation solvers can be found in \texttt{src/DenseLinearAlgebra}.


Any system of building executables in a complex project is necessarily
complex. This is especially true for object-oriented code, as the most
common methods for building executables are designed for use with
procedural (rather than object-oriented) languages. In OOQP, we have
designed a relatively simple process but one that requires some
effort to learn and understand.  Users who intend to develop a
customized solver for a new QP formulation or to replace the linear
algebra subsystem need to understand something of this process, and
this section is aimed primarily at them. Users who do not have an
interest in the details of the build process may safely skip this
section.

OOQP is built by using the GNU version of the standard Unix
\verb-make- utility. GNU \verb-make- is freely and widely available,
yields predictable performance across a wide variety of platforms, and
has a number of useful features absent in many vendor-provided
versions of \verb-make-. In this section, we assume that the user has
a basic understanding of how to write makefiles, which are the files
used as input to the \verb-make- utility.

OOQP uses a \verb-configure- script, generated by the GNU Autoconf
utility, to set machine-dependent variables within the makefiles that
appear in various subdirectories. In the top-level directory,
\texttt{OOQP}, \verb-configure- generates the global makefile
\verb-GNUmakefile- from an input file named \verb-GNUmakefile.in-.
The user who wishes to modify this makefile should alter
\verb-GNUmakefile.in- and then re-run \verb-configure- to obtain a new
\verb-GNUmakefile-, rather than altering \verb-GNUmakefile- directly.
(Users will seldom have cause to alter this makefile or any other file
under the control of Autoconf but should be aware of the fact that
some makefiles are generated in this way.)

All subdirectories of the \texttt{src} that contain C++ code also
contain a file named $\mathtt{Makefile.inc}$.  We give an example of
such a file from the directory \verb-src/QpExample-, which contains an
example problem formulation based directly on (\ref{qp}). In the
\texttt{src/QpExample} directory, the \texttt{Makefile.inc} reads
as follows.
\begin{verbatim}
QPEXAMPLEDIR = $(srcdir)/QpExample

QPEXAMPLEOBJ = \
    $(QPEXAMPLEDIR)/QpExampleData.o \
    $(QPEXAMPLEDIR)/QpExampleVars.o \
    $(QPEXAMPLEDIR)/QpExampleResids.o  \
    $(QPEXAMPLEDIR)/QpExampleDenseLinsys.o \
    $(QPEXAMPLEDIR)/QpExampleDense.o 

qpexample_dense_gondzio_OBJECTS = \
    $(QPEXAMPLEDIR)/QpExampleGondzioDriver.o \
    $(QPEXAMPLEOBJ) \
    $(libooqpgondzio_STATIC) \
    $(libooqpdense_STATIC) $(libooqpbase_STATIC)
\end{verbatim}
% I put a $ here to get the highlighting to work in emacs.
This file contains three makefile variable definitions, specifying the
subdirectory name (\texttt{QPEXAMPLEDIR}), the list of object files
specific to the SVM solver (\texttt{QPEXAMPLEOBJ}), and the full list
of object files that must be linked to create the executable for the
solver (\texttt{qpexample\_dense\_gondzio\_OBJECTS}).  Every module of
OOQP contains a similar \texttt{Makefile.inc} file to define variables
relevant to that module. (Another example is the variable
\texttt{libooqpgondzio\_STATIC}, used in the definition of
\texttt{qpexample\_dense\_gondzio\_OBJECTS}, which is defined in
\texttt{src/QpSolvers/Makefile.inc}.) Note that the variable
\texttt{srcdir} in this example refers to the OOQP source directory
and does not need to be defined in
\texttt{src/QpExample/Makefile.inc}.

Some subdirectories of the \texttt{src} that contain C++ code
also contain a file named $\mathtt{MakefileTargets.inc}$.  This file
defines targets relevant to the build process. An example of such a
file is \verb-src/QpExample/MakefileTargets.inc-, which is as
follows.
\begin{verbatim}
qpexample-dense-gondzio.exe: $(qpexample_dense_gondzio_OBJECTS)
    $(CXX) -o $@ $(CXXFLAGS) $(LDFLAGS) $(LIBS) \
        $(qpexample_dense_gondzio_OBJECTS) $(BLAS) $(FLIBS)
\end{verbatim}
%%
%% $ - an extra dollar sign to balance out the one in verbatim
%% so that the syntaxtic highlighting in emacs works properly.
The \texttt{qpexample-dense-gondzio.exe} target specifies the
dependency of the executable on the object list that was defined
in the corresponding \verb-Makefile.inc- file.

In using \verb-Makefile.inc- and \verb-MakefileTargets.inc- files, we
separate target definitions from variable definitions because
unpredictable behavior can occur if the targets are read before all
variables are defined.

When a user invokes GNU {\tt make} from the {\tt OOQP} directory, the
utility ensures that
\begin{itemize}
  \item all variables defined in files named \verb-Makefile.inc- in
    {\em direct} subdirectories of the {\tt src} directory are
    made available in the build;

  \item all targets defined in similarly located files named
    \verb-MakefileTargets.inc- are also made available;
    
  \item {\em direct} subdirectories of the {\tt src} directory that
    contain a file that is named \verb-Makefile.inc- are placed on the
    path on which to search for header (.h) files.
\end{itemize}
Thus, when the GNU {\tt make} utility is named \verb-gmake-, one may
build the executable \verb|qpexample-dense-gondzio.exe| by typing
\begin{verbatim}
    gmake qpexample-dense-gondzio.exe
\end{verbatim}
from the command line from within the {\tt OOQP} directory.

The makefile system can also be used to perform dependency checking.
Typing 
\begin{verbatim}
    gmake depend
\end{verbatim}
will cause the Unix \verb-makedepend- utility to generate dependency
information for all source files in direct subdirectories of the {\tt
src} directory that contain a file named \verb-Makefile.inc-.  This
dependency information will then be used in the next build to
determine whether source files are up-to-date with respect to their
included header files.

We emphasize that this process works only on direct subdirectories of
the {\tt src} directory. Files named \verb-Makefile.inc- in more
deeply nested subdirectories will not, without extra effort, be
recognized. We deliberately restricted the search to direct
subdirectories of the source directory in order to make the build
process more predictable.

User-defined \verb-Makefile.inc- and \verb-MakefileTargets.inc- need
be no more complicated than the example files given above. Some of the
instances of these files that are included in the OOQP distribution
contain more variables and targets than those shown above because they
need to accomplish additional tasks. Moreover, they may contain
conditional statements to disable certain targets, if these targets
depend on external packages that are not present on the computer at
the time of the build. These advanced issues may be ignored by all but
developers of OOQP.

Finally, we mention that some external packages, such as PETSc,
require specializations to the global makefile. When building
executables that use these packages, one cannot use the default global
makefile {\tt GNUmakefile}. To build the executable
\verb|qpbound-petsc-mehrotra.exe|, for instance, one must type the
following line.
\begin{verbatim}
   gmake -f PetscMakefile qpbound-petsc-mehrotra.exe
\end{verbatim}
We may include other such specialized makefiles in the OOQP
distribution in the future. While inclusion of these files is a minor
inconvenience, we consider it important to isolate changes to the
global makefile in this manner, so that misconfiguration of a certain
package is less likely to cause problems in an unrelated build.

%%% Local Variables: 
%%% mode: latex
%%% TeX-master: "ooqp-userguide"
%%% End: 

\section{Working with the QP Solver}
\label{sec.qp-solver}

In this section, we focus on the top layer of OOQP, the QP solver.

\subsection{Primal-Dual Interior-Point Algorithms}
\label{sec.pdip.algorithms}

We start by giving some details of the primal-dual interior-point
algorithms that are implemented in the {\tt Solver} class in the OOQP
distribution. By design, the code that implements these algorithms is
short, and one can see the correspondence between the code and the
algorithm description below. Therefore, users who want to modify the
basic algorithm will be able to do so after reading this section.

A primal-dual algorithm seeks variables $(x,y,z,s)$ that satisfy the
optimality conditions for the convex quadratic program
\eqnok{qpintro}, introduced in Section~\ref{sec.ooqp-develop-layers} but
repeated here for convenience.
\begin{subequations} \label{rresids}
\beqa
\label{rresids.1}
c + Q x -  A^T y - C^T z & = & 0, \\
\label{rresids.2}
A x - b & = & 0, \\
\label{rresids.3}
C x - s - d & = & 0, \\
\label{rresids.4}
S Z e & = & 0, \\
\label{rresids.5}
s, z & \ge & 0.
\eeqa
\end{subequations}
The complementarity measure $\mu$ defined by
%
\beq 
\label{mu.def} \mu = z^Ts / m_c
\eeq
%
(where $m_c$ is the number of rows in $C$) is important in measuring
the progress of the algorithm, since it measures violation of the
complementarity condition $z^Ts =0$, which is implied by
\eqnok{rresids.4}.  Infeasibility of the iterates with respect to the
equality constraints \eqnok{rresids.1}, \eqnok{rresids.2}, and
\eqnok{rresids.3} also makes up part of the indicator of nonoptimality.

The OOQP distribution contains implementations of two quadratic
programming algorithms: Mehrotra's predictor-corrector method
\cite{Meh92a} and Gondzio's modification of this method that uses
higher-order corrector steps \cite{Gon94d}. (See also
\cite[Chapter~10]{IPPD96} for a discussion of both methods.) These
algorithms have proved to be the most effective methods for linear
programming problems and in our experience are just as effective for
convex quadratic programming. Mehrotra's algorithm can be specified as
follows.

\btab
\> {\bf Algorithm MPC (Mehrotra Predictor-Corrector)} \\
\> Given starting point $(x,y,z,s)$ with $(z,s)>0$, and
parameter $\tau \in [2,4]$; \\
\> {\bf repeat} \\
\>\> Set $\mu = z^Ts / m_c$. \\
\>\> Solve for $(\Dxaff, \Dyaff, \Dzaff, \Dsaff)$: 
\etab
\beq \label{lin.affine}
\bmat{cccc} Q & -A^T & -C^T & 0 \\ 
A & 0 & 0 & 0  \\
C & 0 & 0 & -I \\
0 & 0 & S & Z
\emat \bmat{c} \Dxaff \\ \Dyaff \\ \Dzaff \\ \Dsaff \emat = 
- \bmat{c} r_Q \\ r_A \\ r_C \\ Z S e \emat,
\eeq
\btab
\>\>\> where 
\etab
\begin{subequations} \label{lin.defs}
\beqa
\label{lin.defs.S}
S & =& {\rm diag}(s_1,s_2, \dots,s_{m_c}), \\
\label{lin.defs.Pi}
Z & = & {\rm diag}(z_1,z_2,\dots,z_{m_c}), \\
\label{lin.defs.rQ}
r_Q &=& Qx+c-A^T y-C^T z, \\
\label{lin.defs.rA}
r_A &=& Ax-b, \\
\label{lin.defs.rC}
r_C &=& Cx-s-d.
\eeqa
\end{subequations}
\btab
\>\> Compute $\alpha_{\rm aff}$ to be the largest value in $(0,1]$ such that
\etab
\[
(z,s) + \alpha (\Dzaff,\Dsaff) \ge 0.
\]
\btab
\>\> Set $\mu_{\rm aff} = (z+\alpha_{\rm aff} \Dzaff)^T 
(s+\alpha_{\rm aff} \Dsaff)/m_C$. \\
\>\> Set $\sigma = (\mu_{\rm aff} / \mu)^{\tau}$. \\
\>\> Solve for $(\Dx, \Dy, \Dz, \Ds)$:
\etab
\beq \label{lin.final}
\bmat{cccc} Q & -A^T & -C^T & 0 \\ 
A & 0 & 0 & 0  \\
C & 0 & 0 & -I \\
0 & 0 & S & Z
\emat \bmat{c} \Dx \\ \Dy \\ \Dz \\ \Ds \emat =
- \bmat{c} r_Q \\ r_A \\ r_C \\ Z S e  - \sigma \mu e + 
\D Z^{\rm aff} \D S^{\rm aff} e \emat,
\eeq
\btab
\>\>\> where $\D Z^{\rm aff}$ and $\D S^{\rm aff}$ are defined in an obvious way. \\
\>\> Compute $\alpha_{\rm max}$ to be the largest value in $(0,1]$ such that
\etab
\[
(z,s) + \alpha (\Dz,\Ds) \ge 0.
\]
\btab
\>\> Choose $\alpha \in (0,\alpha_{\rm max})$ according to 
Mehrotra's step length heuristic. \\
\>\> Set
\etab
\[
(x,y,z,s) \leftarrow (x,y,z,s) + \alpha (\Dx, \Dy, \Dz, \Ds).
\]
\btab
\> {\bf until }  the convergence or infeasibility test is satisfied.
\etab

The direction obtained from \eqnok{lin.final} can be viewed as an
approximate second-order step toward a point $(x^+,y^+,z^+,s^+)$ at
which the conditions \eqnok{rresids.1}, \eqnok{rresids.2}, and
\eqnok{rresids.3} are satisfied and, in addition, the pairwise products
$z_i^+ s_i^+$ are all equal to $\sigma \mu$. The heuristic for
$\sigma$ yields a value in the range $(0,1)$, so the step usually
produces a reduction in the average value of the pairwise products
from their current average of $\mu$.

Gondzio's approach~\cite{Gon94d} follows the Mehrotra algorithm in its
computation of directions from \eqnok{lin.affine} and
\eqnok{lin.final}. It may then go on to enhance the search direction
further by solving additional systems similar to \eqnok{lin.final},
with variations in the last $m_C$ components of the right-hand side.
Successive corrections attempt to increase the steplength $\alpha$
that can be taken along the final direction, and to bring the pairwise
products $s_i z_i$ whose values are either much larger than or much
smaller than the average into closer correspondence with the
average. The maximum number of corrected steps we calculate is
dictated by the ratio of the time taken to factor the coefficient
matrix in \eqnok{lin.final} to the time taken to use these factors to
produce a solution for a given right-hand side. When the marginal cost
of solving for an additional right-hand side is small relative to the
cost of a fresh factorization, and when the corrections appear to be
improving the quality of the step significantly, we allow more
correctors to be calculated, up to a limit of $5$.

The algorithms implemented in OOQP use the step length heuristic
described in Mehrotra~\cite[Section~6]{Meh92a}, modified slightly to
ensure that the same step lengths are used for both primal and dual
variables.

\subsection{Monitoring the Algorithm: The {\tt Monitor} Class}
\label{sec.monitor}

OOQP can be used both for solving a variety of stand-alone QPs and for
solving QP subproblems as part of a larger algorithm.  Different
termination criteria may be appropriate to each context. For a simple
example, the criteria used to declare success in the solution of a
single QP would typically be more stringent than the criteria for a QP
subproblem in a nonlinear programming algorithm, in which we can
afford some inexactness in the solution.  Accordingly, we have
designed OOQP to be flexible as to the definition and application of
termination criteria, and as to the way in which the algorithm's
progress is monitored and communicated to the user. In some instances,
a short report on each interior-point iteration is desirable, while in
others, silence is more appropriate.  In OOQP, an abstract
\texttt{Monitor} class monitors the algorithm's progress, while an
abstract \texttt{Status} class tests the termination conditions. We
describe the \texttt{Monitor} class in this section, and the
\texttt{Status} class in Section~\ref{sec.status} below.

Our design assumes that each algorithm in the QP solver layer of the
code has its own natural way of monitoring the algorithm and testing
termination. Accordingly, the two derived {\tt Solver} classes in the
OOQP distribution each contain a {\tt defaultMonitor} method to print
out a single line of information to the standard output stream at each
iteration, along with a suitable message at termination of the
algorithm. The prototype of this method is as follows.
\begin{verbatim}
void Solver::defaultMonitor( Data * data, Variables * vars,
                                    Residuals * resids,
                                    int i, double mu, 
                                    int status_code, int stage )
\end{verbatim}
The {\tt data} argument contains the problem data, while {\tt vars}
and {\tt resids} contain the values of the variables and residuals at
the current iterate, which together depict the status of the
algorithm. (See Sections~\ref{sec.ooqp-develop-layers} and
\ref{sec.new-qp-formulation} for further information about these
objects.)  The variable \texttt{i} is the current iteration number and
\texttt{mu} is the complementarity measure \eqnok{mu.def}. The integer
{\tt status\_code} indicates the status of the algorithm at
termination, if termination has occurred; see Section~\ref{sec.status}
below. The {\tt stage} argument indicates to {\tt defaultMonitor} what
type of information it should print.  In our implementations, the
values {\tt stage=0} and {\tt stage=1} cause the routine to print out
a single line containing iteration number, the value of $\mu$, and the
residual norm. The value {\tt stage=1} is used after termination has
occurred, and additionally causes a message about the termination
status to be printed.

One mechanism available to the user who wishes to alter
the monitoring procedure is to create a new subclass of {\tt Solver}
that contains an implementation of {\tt defaultMonitor} that overrides
the existing implementation. This is the simplest way to proceed and
will suffice in many circumstances.  However, it has a disadvantage for
users who work with several different implementations of {\tt
Solver}---versions that implement different primal-dual algorithms,
for instance, or are customized to different applications---in that
the new monitoring routine cannot be shared among the different QP
solvers. A subclass of each QP solver that contains the overriding
implementation of {\tt defaultMonitor} would need to be created,
resulting in a number of new leaves on the class tree. A second
disadvantage is that some applications may require several monitor
processes to operate at once, for example, one process like the {\tt
defaultMonitor} described above that writes minimal output to standard
output, and another process that writes more detailed information to a
log file. It is undesirable to create a new {\tt Solver} subclass for
each different set of monitor requirements.

In OOQP, we choose delegation, rather than subclassing, as our
mechanism for customizing the monitor process. Delegation is a
technique in which the responsibility for taking some action normally
associated with an instance of a given class is delegated to some
other object. In our case, although the {\tt Solver} class would
normally be responsible for displaying monitor information, we
delegate responsibility to an associated instance of the
\texttt{Monitor} class. The {\tt Solver} class contains methods for
establishing its {\tt defaultMonitor} method as one of the monitor
procedures called by the code and for adding monitor
procedures supplied by the user.


% when the {\tt Solver} object is associated with the {\tt Monitor}
% object.

The abstract definition of the {\tt Monitor} class can be found in the
OOQP distribution at {\tt src/Abstract/OOQPMonitor.h}, along with
the definitions of several subclasses. 
% Instances of \texttt{Monitor} are little more than glorified subroutines. 
The only method of interest in the \texttt{Monitor} class is the
\texttt{doIt} method, which causes the object to perform the operation
that is its sole reason for being. Making these objects instances of a
class rather than subroutines tends to be more natural in the C++
language and makes it far simpler to handle any state information that
instances of \texttt{Monitor} may wish to keep between calls to
\texttt{doIt}.

The \texttt{doIt} method has the following prototype, which is
identical to the {\tt defaultMonitor} method described above.
\begin{verbatim}
void OoqpMonitor::doIt( Solver * solver, Data * data, 
                        Variables * vars, Residuals * resids,
                        int i, double mu, 
                        int status_code, int stage );
\end{verbatim}
Users who wish to implement their own monitor procedure should create
a subclass of {\tt OOQPMonitor}, for example by making the following
definition:
\begin{verbatim}
class myMonitor : public OOQPMonitor {
public:
  virtual void doIt( Solver * solver,  Data * data,
                     Variables * vars, Residuals * resids,
                     int i, double mu,
                     int status_code, int level );
};
\end{verbatim}
and then implementing their own version of the {\tt doIt}
method. Their code that creates the instance of the {\tt Solver} class
and uses it to solve the QP should contain the following code
fragments:
\begin{verbatim}
OoqpMonitor * usermon = new myMonitor;
...
qpsolver->monitorSelf();
qpsolver->addMonitor( usermon );
\end{verbatim}
The first statement creates an instance of the subclass {\tt
  myMonitor}. The second and third statements should appear after the
instance {\tt qpsolver} of the {\tt Solver} class has been created but
before the method {\tt qpsolver->solve()} has been invoked. The call
to \texttt{monitorSelf} statement ensures that the {\tt
  defaultMonitor} method is invoked at each interior-point iteration,
while the call to \texttt{addMonitor} ensures that the user-defined
monitor is also invoked. Users who wish to invoke only their own
monitor procedure and not the {\tt defaultMonitor} method can omit the
second statement. The solver is responsible for deleting any monitors
  give to it via the \texttt{addMonitor} method.

The default behavior for an instance of \texttt{Solver} is to
display no monitor information.



\subsection{Checking Termination Conditions: The {\tt Status} Class}
\label{sec.status}

In OOQP, the \texttt{defaultStatus} method of the {\tt Solver} class
normally handles termination tests. However, OOQP allows delegation of
these tests to an instance of the {\tt Status} class, in much the same
way as the monitor procedures can be delegated as described above.
Before describing how to replace the OOQP termination tests, let us
describe the termination tests that OOQP uses by default.

The \texttt{defaultStatus} method of the {\tt Solver} class uses
termination criteria similar to those of PCx~\cite{PCx99}. To discuss
these criteria, we again refer to the problem formulation
\eqnok{qpintro} (discussed in Section~\ref{sec.pdip.algorithms}) and
use $(x^k,y^k,z^k,s^k)$ to denote the primal-dual variables at
iteration $k$, and $\mu_k \defeq (z^k)^T s^k/m_C$ to denote the
corresponding value of $\mu$.  Let $r_Q^k$, $r_A^k$, and $r_C^k$ be
the values of the residuals at iteration $k$, and let $\mbox{gap}_k$
be the duality gap at iteration $k$, which may be defined for
formulation \eqnok{qp} by the formula \eqnok{gapk} below. We define
the quantity $\phi_k$ as follows,
\[
\phi_k \defeq \frac{\| (r_Q^k,r_A^k,r_C^k) \|_{\infty} + \mbox{gap}_k}{\| (Q,A,C,c,b,d)\|_{\infty}},
\]
where the denominator is simply the element of largest magnitude in
all the data quantities that define the problem \eqnok{qp}. Note that
$\phi_k=0$ if and only if $(x^k,y^k,z^k,s^k)$ is optimal.

Given parameters ${\tt tol}_{\mu}$ and ${\tt tol}_r$ (both of which
have default value $10^{-8}$), we declare the termination status to be
{\tt SUCCESSFUL\_TERMINATION} when
\beq \label{terminate:success}
\mu_k \le {\tt tol}_{\mu}, \gap
\| (r_Q^k, r_A^k, r_C^k) \|_{\infty} \le 
{\tt tol}_r \| (Q,A,C,c,b,d)\|_{\infty}.
\eeq
We declare the status to be {\tt INFEASIBLE} if
\beq \label{terminate:infeas}
\phi_k > 10^{-8} \;\; \mbox{and} \;\; 
\phi_k \ge 10^4 \min_{0 \le i \le k} \phi_i.
\eeq
(In fact, since this is not a foolproof test of infeasibility, the
true meaning of this status is ``probably infeasible.'')  Status {\tt
UNKNOWN} is declared if the algorithm appears to be making
unacceptably slow progress, that is,
\beq \label{terminate:unknown1}
k \ge 30 \;\; \mbox{and} \;\; \min_{0 \le i \le k} \phi_i \ge
\frac12 \min_{1 \le i \le k-30} \phi_i,
\eeq
or if the ratio of infeasibility to the value of $\mu$ appears to be 
blowing up, that is,
\begin{subequations} \label{terminate:unknown2}
\beqa
& \| (r_Q^k, r_A^k, r_C^k) \|_{\infty} >
{\tt tol}_r \| (Q,A,C,c,b,d)\|_{\infty} \\
& \mbox{and} \;\; {\| (r_Q^k,r_A^k,r_C^k)\|_{\infty}} / {\mu_k} \ge 10^8
{\| (r_Q^0,r_A^0,r_C^0)\|_{\infty}} / {\mu_0}.
\eeqa
\end{subequations}
We declare status {\tt MAX\_ITS\_EXCEEDED} when the number of
iterations exceeds a specified maximum; the default is 100.  If none
of these conditions is satisfied, we declare the status to be {\tt
NOT\_FINISHED}.

Users who wish to alter the termination test may
simply create a subclass of {\tt Solver} with their own implementation
of {\tt defaultStatus}. Alternatively, they may create a subclass of
the {\tt Status} class, whose abstract definition can be found in the
file {\tt src/Abstract/Status.h}. The sole method in the {\tt Status}
class is \texttt{doIt}, which has the following prototype.
\begin{verbatim}
int Status::doIt(  Solver * solver, Data * data, 
                   Variables * vars, Residuals * resids,
                   int i, double mu, int stage );
\end{verbatim}
The parameters to the \texttt{doIt} method have the same meaning as
the correspondingly named parameters of the \texttt{OOQPMonitor::doIt}
method. The return value of the \texttt{Status::doIt} method
determines whether the algorithm continues or terminates. The possible
values that may be returned are as follows.
\begin{verbatim}
enum TerminationCode 
{
  SUCCESSFUL_TERMINATION = 0,
  NOT_FINISHED,
  MAX_ITS_EXCEEDED,
  INFEASIBLE,
  UNKNOWN
};
\end{verbatim}
The meanings of these return codes in the {\tt defaultStatus} method
are described above. Users are advised to assign similar meanings in
their specialized implementation. 

Unlike the case of monitor procedures, it does not make sense to have
multiple status checks in operation during execution of the
interior-point algorithm; exactly one such check is required.  Users
who wish to use the {\tt defaultStatus} method supplied with the OOQP
distributions need do nothing; the default behavior of an instance of
the {\tt Solver} class is to call this method. Users who wish to
supply their own method can create their own subclass of the {\tt
  Status} class as follows.
\begin{verbatim}
class myStatus : public Status {
public:
        virtual void doIt(  Solver * solver, Data * data, 
                            Variables * vars, Residuals * resids,
                            int i, double mu, int stage );
};
\end{verbatim}
Then, they can invoke the {\tt useStatus} method after creating their
instance of the {\tt Solver} class, to indicate to the solver object
that it should use the user-defined status-checking method. The
appropriate lines in the driver code would be similar to the
following.
\begin{verbatim}
MyStatus * userstat = new myStatus;
...
qpsolver->useStatus( userstat );
\end{verbatim}
The solver is responsible for deleting any \texttt{Status} objects
given to it via the \texttt{useStatus} method.



%%% Local Variables: 
%%% mode: latex
%%% TeX-master: "ooqp-userguide"
%%% End: 


\section{Creating a New QP Formulation}

\label{sec.new-qp-formulation}

Users who wish to construct a solver for a class of QPs with a
particular structure not supported in the OOQP distribution may
consider using the framework to build a new solver that represents and
manipulates the problem data and variables in an economical, natural,
and efficient way. In this section, we describe the major classes that
must be implemented in order to develop a solver for a new problem
formulation.

Most of the effort in developing a customized solver for a new class
of structured QPs is in reimplementing the classes in the problem
formulation layer.  As described in
Section~\ref{ooqp-develop-overview}, this layer consists of five main
classes---{\tt Data}, {\tt Variables}, {\tt Residuals}, {\tt
LinearSystem}, and {\tt ProblemFormulation}---that contain data
structures to store the problem data, variables, and residuals, and
methods to perform the operations that are required by the
interior-point algorithms. 

As discussed in Section~\ref{sec.pdip.algorithms}, the core algebraic
operation in an interior-point solver is the solution of a Newton-like
system of linear equations. For formulation~\eqnok{qp}, the general
form of this system is as follows
\beq \label{lin.general}
\bmat{cccc} Q & -A^T & -C^T & 0 \\
A & 0 & 0 & 0  \\
C & 0 & 0 & -I \\
0 & 0 & S & Z \emat \bmat{c} \Dx \\ \Dy \\ \Dz \\ \Ds \emat = -
\bmat{c} r_Q \\ r_A \\ r_C \\ r_{z,s} \emat,
\eeq 
%
where $r_Q$, $r_A$, and $r_C$ are defined in
equations~\eqnok{lin.defs.rQ}, \eqnok{lin.defs.rA},
and~\eqnok{lin.defs.rC}, and $r_{z,s}$ is chosen in a variety of ways, as
described in Section~\ref{sec.qp-solver}.  Most of the objects that
populate a problem formulation layer can be found in this system. The
\texttt{Variables} in formulation \eqnok{qp} break down naturally into
four components $x$, $y$, $z$, and $s$. Likewise, there are naturally
four components to the \texttt{Residuals} of this formulation. For
other problem formulations, such as SVM \eqnok{svm-formulation}, this
partitioning of the variables is not natural, and a scheme more suited
to the particular formulation is used instead. 
However, to focus our discussion of the implementation of the problem
formulation layer in this section, we will continue to refer to the
particular formulation \eqnok{qp} and the system \eqnok{lin.general}.
The implementations of \eqnok{qp} discussed in this section may be
found in the OOQP distribution in directory \texttt{src/QpExample}.

In reimplementing the problem formulation layer for a new QP
structure, it may be helpful to make use of the classes from the {\em
linear algebra layer}. As mentioned in
Section~\ref{ooqp-develop-overview}, this layer contains classes for
storing and operating on dense matrices, sparse matrices, and
vectors. These classes can be used as building blocks for implementing
the more complex storage schemes and arithmetic operations needed in
the problem formulation layer.

We first elaborate on the use of the linear algebra layer and then
describe in some detail the process of implementing the five classes
in the problem formulation layer.

\subsection{Linear Algebra Operations}

Most implementations of the problem formulation layer that appear in
the OOQP distribution (the {\tt QpGen}, {\tt QpExample}, {\tt
QpBound}, and {\tt Huber}, and {\tt Svm} implementations) all are
built using the objects in OOQP's linear algebra layer. The classes in
this layer represent objects such as matrices and vectors, and they
provide methods that are especially useful for developing interior
point QP solvers.  By basing our problem formulation layer on the
abstract operations of the linear algebra layer we gain another
significant advantage: we can use the same problem formulation code
for several quite varied representations of vectors and matrices.  For
instance, the implementation of the problem formulation layer for QPs
with simple bounds is independent of whether the Hessian matrix is
represented as a dense array on a single processor or as a sparse
array distributed across several processors.

Use of OOQP's linear algebra layer in implementing the problem
formulation layer is not mandatory.  Users are free to define their
own matrix and vector data structures and implement their own linear
algebra operations (inner products, saxpys, factorizations, and so on)
without referring to OOQP's linear algebra objects at all.  The
authors of OOQP recognize that there is a learning curve associated
with the use of the abstract operations in OOQP's linear algebra
objects and that the implementation might proceed more quickly if
users define their own linear algebra in terms of concrete operations
on concrete data.

For maximum effectiveness, we recommend a compromise approach. While
the base classes for our linear algebra layer are defined only in
terms of abstract operations, several of the classes (such as {\tt
SimpleVector}) may also be used concretely.  Users can start by
defining their problem formulation in terms of these simple classes
but define their own concrete operations on the data. Later, they can
replace their concrete operations by the abstract methods supplied
with these classes. Finally, having gained proficiency in the use of
these classes, they may then replace the entire class with a more
appropriate one.  Section~\ref{sec.using-linear-algebra} is a short
tutorial on the linear algebra layer that can be consulted by those
who wish to use the layer in this way.

\subsection{Specializing the Problem Formulation Layer}

We now detail how to implement the various classes in the problem
formulation layer.

\subsubsection{Specializing {\tt Data}}
\label{sec:dataclass}

The purpose of the {\tt Data} class is to store the data defining the
problem, in some appropriate format, to provide methods for performing
operations with the data matrices (for example, matrix
multiplications or insertion of problem matrices into the larger
matrices of the form \eqnok{lin.affine} or \eqnok{lin.final}), for
calculating some norm of the data, for filling the data structures
with problem data (read from a file, for instance, or passed from a
modeling language or MATLAB), for printing the data, and for
generating random problem instances for testing and benchmarking
purposes.

Since both the data structures and the methods implemented in {\tt
  Data} depend so strongly on the structure of the problem, the parent
class is almost empty.  It includes only two pure virtual functions,
{\tt datanorm} (of type {\tt double}) and {\tt print}, whose
implementation {\em must} appear in any derived classes.

A derived class of {\tt Data} for the formulation \eqnok{qp} in which
the problem data is dense would include storage for the vectors $c$,
$b$, and $d$ as arrays of doubles; storage for $A$ and $C$ as
two-dimensional arrays of doubles; and storage for the lower triangle
of the symmetric matrix $Q$. In our implementation of the derived
class \texttt{QpExampleData}, we have provided methods for multiplying
by the matrices $Q$, $A$, and $C$ and for copying the data into a
larger structures such as the matrix in~\eqnok{lin.general}. We find
it convenient to provide methods like this for manipulating the data
in our \texttt{QpExampleData} class, rather than having code from
other problem formulation classes manipulate the data structures
directly; the extra generality that the added layer of encapsulation
affords has sometimes proven useful.
% We discuss in Section~\ref{sec:linsysclass} the possible techniques
% for solving \eqnok{lin.equiv.1}, but note for now that the derived
% {\tt Data} class for this case would contain methods for placing $Q$,
% $A$, and $C$ into the data structure for this system (which is stored
% in the {\tt LinearSystem} class). Obviously these methods must take
% account of the format used to store the large matrix; for example,
% whether it is stored without regard to symmetry in Harwell-Boeing
% format, or whether it is stored in some more compact symmetric format.


Consider now the two pure virtual functions {\tt datanorm} and {\tt
print}. One reasonable implementation of {\tt datanorm} for the
formulation \eqnok{qp} would simply return the magnitude of of the
largest element in the matrices $Q$, $A$, and $C$, and the vectors
$c$, $b$, and $d$ that define \eqnok{qp}.  The implementation of {\tt
print} might print the data objects $Q$, $A$, $C$, $c$, $b$, and $d$
to standard output in some useful format. Although not compulsory, we
might also define a routine {\tt datarandom} to generate an instance
of \eqnok{qp}, given user-defined dimensions $n$, $m_A$, and $m_C$,
and possibly a desired level of sparsity for the matrices.  Naturally,
this method should take care that $Q$ is positive semidefinite.

The derived {\tt Data} class might also contain one or more
implementations of a {\tt datainput} method that allow the user to
define the problem data. We could, for instance, have one implementation
of {\tt datainput} that reads the data in some simple format from
ascii files and another implementation that reads a file in MPS
format, appropriately extended for quadratic programming (Maros and
M\'esz\'aros~\cite{MarM99}). Since the MPS format allows for bounds
and for constraints of the form $l_c \le C x \le u_c$, the latter
implementation generally would need to perform transformations to pose
the problem in the form \eqnok{qp}. (The data from a MPS file is more
naturally represented by our ``general'' QP formulation~\eqnok{qpgen}.)

% Numerous other methods can be added to the derived class, over and
% above those defined in the parent {\tt Data} class. As we discuss in
% later sections, our implementation of the solver for \eqnok{qp}
% include auxiliary methods that construct the linear system to be
% solved at each iteration; that perform multiplications of given
% vectors by $A$, $A$, $C$, $A^T$, and $C^T$; that ``get'' certain data
% elements (for example, {\tt getM1} returns the number of rows in $A$).
% These methods are specific to the problem structure \eqnok{qp}, so are
% not defined in the parent class {\tt Data}.

\subsubsection{Specializing {\tt Variables}}
\label{sec:variablesclass}

Instances of {\tt Variables} class store the problem variables
($(x,y,z,s)$ in the case of \eqnok{qp}) in whatever format is
appropriate to the problem structure. The class includes a variety of
methods essential in the implementation of Algorithm MPC. Most of them
defined as pure virtual functions, because they strongly depend on the
structure of the problem.

We now sketch the main methods for the {\tt Variables} class,
illustrating each one by specifying its implementation for the
formulation \eqnok{qp}.

\begin{description}
\item[] {\tt mu}: Calculate the complementarity gap: $\mu = z^Ts/m_C$.
  
\item[] {\tt mustep}: Calculate the complementarity gap that would be
obtained from a step of length $\alpha$ along a specified direction
from the current point. For \eqnok{qp}, given the search direction
$(\Dx, \Dy, \Dz, \Ds)$ (supplied in an argument of type {\tt
Variables}) and a positive scalar $\alpha$, this method would
calculate
\[
(z+\alpha \Dz)^T (s+\alpha \Ds) / m_C.
\]

\item[] {\tt negate}: Multiply the current set of variables by $-1$. For
\eqnok{qp}, we would replace $(x,y,z,s)$ by $-(x,y,z,s)$.
  
\item[] {\tt saxpy}: Given another set of variables and a scalar,
perform a saxpy operation with the current set of variables. For
\eqnok{qp}, we would pass a second instance of a {\tt Variables} class
containing $(x',y',z',s')$, together with the scalar $\alpha$ as
arguments, and perform the replacement
\[
(x,y,z,s) \leftarrow  (x,y,z,s) + \alpha (x',y',z',s').
\]

\item[] {\tt stepbound}: Calculate the longest step in the range
$[0,1]$ that can be taken from the current point in a specified
direction without violating nonnegativity of the complementary
variables. For \eqnok{qp}, the argument would be the direction
$(x',y',z',s')$ (stored in another instance of the{\tt Variables}
class), and this function would return the largest value of $\alpha$
in $[0,1]$ such that the condition $(z + \alpha z', s + \alpha s') \ge
0$ is satisfied.
  
\item[] {\tt findBlocking}: Similar to {\tt stepbound} but returns
additional information. Besides returning the maximum step $\alpha$ in
the range $(0,1]$ that can be taken without violating the appropriate
nonnegativity constraint, the method indicates whether a primal or
dual variable was the ``blocking'' variable (the one that will violate
nonnegativity if the step $\alpha$ is any longer) by setting its last
argument to $1$ for a primal blocking variable, to $2$ for a dual
blocking variable, and to $0$ if a full steplength $\alpha = 1$ can be
taken without violating nonnegativity. In its second argument, the
method returns the component of the primal variable vector that
corresponds to the blocking index, while in its third argument, the
method returns the same component of the primal {\em step} vector. In
its fourth and fifth arguments, it returns the corresponding
components of the dual variable vector and the dual step vector,
respectively. To illustrate this functionality, suppose in the case of
\eqnok{qp} that the step bound is $\alpha$ and the blocking variable
is the $i$th primal variable; that is, $s_i + \alpha s'_i = 0$, while
$(z+ \alpha z', s+\alpha s') \ge 0$. Then the final argument of {\tt
findBlocking} returns $1$, while the second through fifth arguments
return the real numbers $s_i$, $s'_i$, $z_i$, and $z'_i$,
respectively. The return value of the method itself would be $\alpha$.

When both a primal and a dual index are ``blocking,'' the method
reports the dual variable, by setting the final argument to $2$ and
reporting the components corresponding to the dual index. Subject to
the latter condition, when there is a tie between different indices,
the smaller index is reported.
  
\item[] {\tt interiorPoint}: Set all components of the complementary
  variables to specified positive constants $\alpha$ and $\beta$. In
  the case of \eqnok{qp}, we would set $s \leftarrow \alpha e$ and
  $z \leftarrow \beta e$, where $e$ is the vector whose elements are
  all $1$.
  
\item[] {\tt shiftBoundVariables}: Add specified positive constants
  $\alpha$ and $\beta$ to the complementary variables. For \eqnok{qp},
  this method would perform the replacements $s \leftarrow s + \alpha
  e$ and $z \leftarrow z + \beta e$.
  
\item[]{\tt print}: Print the variables in some intelligible
  problem-dependent format.
  
\item[] {\tt copy}: Copy the data from one instance of the {\tt
    Variables} class into another.
  
\item[] {\tt onenorm, infnorm}: Compute the $\ell_1$ and
  $\ell_{\infty}$ norms of the variables. For \eqnok{qp}, these
  quantities would be $\| (x,y,z,s) \|_1$ and $\|
  (x,y,z,s) \|_{\infty}$, respectively.

\end{description}


The usefulness of some of these methods in implementing Algorithm MPC
is obvious. For instance, {\tt saxpy} is used to take a step along the
eventual search direction; {\tt stepbound} is used to compute
$\alpha_{\rm aff}$ and $\alpha_{\rm max}$; {\tt mustep} is used
to compute $\mu_{\rm aff}$. The methods {\tt interiorPoint} and {\tt
  shiftBoundVariables} can be used in the heuristic to determine the
starting point, while {\tt findBlocking} plays an important role in
Mehrotra's heuristic for determining the step length.


\subsubsection{Specializing {\tt Residuals}}
\label{sec:residualsclass}

The {\tt Residuals} class calculates and stores the quantities that
appear on the right-hand side of the linear systems that are solved at
each iteration of the primal-dual method. These residuals can be
partitioned into two fundamental categories: the components arising
from the linear equations in the KKT conditions, and the components
arising from the complementarity conditions. For the formulation
\eqnok{qp}, the components $r_Q$, $r_A$, and $r_C$ (which arise from
KKT linear equations \eqnok{lin.defs.rQ}, \eqnok{lin.defs.rA}, and
\eqnok{lin.defs.rC}) belong to the former class, while $r_{z,s}$
belongs to the latter.  As above, we describe the roles of the main
methods in the {\tt Residuals} class with reference to the formulation
\eqnok{qp}.

\begin{description}
\item[] {\tt calcresids}: Given a {\tt Data} object and a {\tt
    Variables} object, calculate the residual components arising from
  the KKT linear equations. For \eqnok{qp}, this method calculates
  $r_Q$, $r_A$, and $r_C$ using the formulae \eqnok{lin.defs.rQ},
  \eqnok{lin.defs.rA}, and \eqnok{lin.defs.rC}, respectively.
  
\item[] {\tt dualityGap}: Calculate the duality gap, which we define
for the formulation \eqnok{qp} as follows:
\beq \label{gapk}
\mbox{gap}_k \defeq  (x^k)^T  Q x^k - b^T y^k + c^T x^k - d^T z^k.
\eeq
See the discussion below for guidance in formulating an expression for
this parameter.
  
\item[] {\tt residualNorm}: Calculate the norm of the components
  arising from the KKT linear equations. For \eqnok{qp}, this method
  returns $\| (r_Q, r_A, r_C) \|$ for some norm $\| \cdot \|$.
  
\item[] {\tt clear\_r1r2}: Zero the components arising from the KKT
  linear equations. (Gondzio's method requires the solution of linear
  equations in which these residual components are replaced by zeros.)
  
\item[] {\tt clear\_r3}: Set the complementarity components to zero.
  In the case of \eqnok{qp}, for which the general form of the linear
  system is \eqnok{lin.general}, this operation sets $r_{z,s}
  \leftarrow 0$. (This operation is needed only in Gondzio's
  algorithm.)
 
\item[] {\tt add\_r3\_xz\_alpha}: Given a scalar $\alpha$ and a {\tt
    Variables} class, add a complementarity term and a constant to
  each of the complementarity components of the residual vector. For
  \eqnok{qp}, given variables $(x,y,z,s)$, we would set
\[
r_{z,s} \leftarrow r_{z,s} + Z S e + \alpha e,
\]
where $Z$ and $S$ are the diagonal matrices constructed from the
$z$ and $s$ variables.

\item[] {\tt set\_r3\_xz\_alpha}: As for {\tt add\_r3\_xz\_alpha}, but
  overwrite the existing value of $r_{z,s}$; that is, set $r_{z,s}
  \leftarrow Z S e + \alpha$.
  
\item[] {\tt project\_r3}: Perform the projection operation used in
Gondzio's method on the $r_{z,s}$ component of the residual, using the
scalars $\rho_{\rm min}$ and $\rho_{\rm max}$.

\end{description}

As discussed in Section~\ref{sec.status}, the {\tt residualNorm} and
{\tt dualityGap} functions are used in termination and infeasibility
tests.  Users familiar with optimization theory will recognize the
concept of the duality gap and will also recognize that the formula
$x^TQx - b^Ty + c^Tx - d^Tz$ used in \eqnok{gapk} is one of a number
of expressions that are equivalent when the residuals $r_Q$, $r_A$,
and $r_C$ are all equal to zero. One such equivalent expression is the
formula $s^T z$, used in the definition of $\mu$ in the {\tt
Variables} class. We find it useful, however, to use a definition of
the duality gap from which the slack variables have been eliminated
and all the linear equalities in the KKT conditions have been taken
into account. Such a definition can be obtained by starting with the
definition of $\mu$ and successively substituting from each of the KKT
conditions. For the case of \eqnok{qp}, we start with $s^Tz$,
substitute for $s$ from the equation $Cx-s-d=0$ (see \eqnok{resids.3})
to obtain $z^T(Cx-d)$, then substitute for $C^Tz$ from $c + Q x - A^T
y - C^T z = 0$ (see \eqnok{resids.1}) to obtain $c^Tx + x^TQx - x^TAy
- d^Tz$, and finally substitute for $Ax$ from $Ax-b=0$ (see
\eqnok{resids.2}) to obtain the final expression.

% While any of these equivalent expressions will be
% effective for the termination test, the most useful expression for
% detecting infeasibility will depend on the problem formulation. Users
% defining their own problem formulation may obtain a useful expression
% for the duality gap by starting with an expression such as $s\T z$ and
% applying the following general process.
% 
% First, identify which variables in the expression $s\T z$ are slack
% variables, and substitute an equivalent expression for these variables
% based on the residual equations. A slack variable appears in the
% residual equations with the identity as its coefficient.  In this
% example $s$ is the slack variable in the equation $r_C = C x - d - s =
% 0$.  The expression for the gap in our sample formulation after
% substituting for $s$ is
% \[
%  x\T C\T z - d\T z. 
% \]
% Next, substitute expressions derived from any remaining residual equations
% into the expression for the gap, using each residual equation exactly
% once. For instance, continuing the derivation of the gap for the
% example formulation, we may substitute $C\T z = c + Q x - A\T y$ 
% to obtain
% \[
% c\T x + x\T Q x - x\T A\T y - d\T z
% \]
% and then substitute $A x = b$ to obtain the final expression for the gap
% \[
% \mbox{gap} \defeq  x^T  Q x - b^T y + c^T x - d^T z.
% \]
% One should strive to use as many non-slack variables and constant
% vectors as possible in the duality gap.

In Algorithm MPC, the method
{\tt set\_r3\_xz\_alpha} is called with the current {\tt Variables}
and $\alpha=0$ to calculate the right-hand side for the affine-scaling
system \eqnok{lin.affine}. Once $\sigma$ has been determined and the
affine-scaling step is known, {\tt add\_r3\_xz\_alpha} is called with
$\alpha = -\sigma \mu$ and the {\tt Variables} instance that contains
the affine-scaling step, to add the necessary terms to the $r_{z,s}$
component to obtain the system \eqnok{lin.final}. 

% The purpose of the other methods in the algorithm is fairly obvious.
 
\subsubsection{Specializing {\tt LinearSystem}}
\label{sec:linsysclass}

As mentioned above, major algebraic operations at each interior-point
iteration are solutions of linear systems to obtain the predictor and
corrector steps. For the formulation \eqnok{qp}, these systems have
the form \eqnok{lin.general}.  Such systems need to be solved two to
six times per iteration, for different choices of the right-hand side
components but the same coefficient matrix.  Accordingly, it makes
sense to logically separate the \texttt{factor} method that
operates only on the matrix and the {\tt
  solve} method that operates on a specific right-hand side.

We use the term ``factor'' in a general sense, to indicate the part of
the solution process that is {\em independent of the right-hand side}.
The {\tt factor} method could involve certain block-elimination
operations on the coefficient matrix, together with an $LU$, $LDL^T$,
or Cholesky factorization of a reduced system. Alternatively, when we
use an iterative solver, the {\tt factor} operation could involve
computation of a preconditioner.  The {\tt factor} class may need to
include storage---for a permutation matrix, for triangular factors of
a reduced system, or for a preconditioner---for use in subsequent {\tt
solve} operations.  We use the term ``solve'' to indicate that part of
the solution process depends on the specific right-hand side. Usually,
the results of applying methods from the {\tt factor} class are used
to facilitate or speed the process. Depending on the algorithm we
employ, the {\tt solve} method could involve triangular
back-and-forward substitutions, matrix-vector multiplications,
applications of a preconditioner, and/or permutation of vector
components.

Both {\tt factor} and {\tt solve} are pure virtual functions; their
implementation is left to the derived class because they depend entirely
on the problem structure.  For problems with special structure, the
{\tt factor} method is the one in OOQP that gives the most scope for
exploitation of the structure and for computational savings over naive
strategies. The SVM formulation is one case in which an appropriate
implementation of the {\tt factor} class yields significant savings
over an implementation that is not aware of the structure. Another
instances in which an appropriate implementation of {\tt factor} can
produce large computational savings include the case in which $Q$,
$A$, and $C$ have a block-diagonal structure, as in optimal control
problems, allowing \eqnok{lin.equiv.1} to be reordered and solved with
either a banded matrix factorization routine or a discrete Riccati
substitution (Rao, Wright, and Rawlings~\cite{RaoWR97}).

We now describe possible implementations of {\tt factor} for the
formulation \eqnok{qp}.  Direct factorization of the matrix in
\eqnok{lin.general} is not efficient in general as it ignores the
significant structure in this system---the fact that $S$ and $Z$ are
diagonal and the presence of a number of zero blocks. Since the
diagonal elements of $Z$ and $S$ are strictly positive, we can do a
step of block elimination to obtain the following equivalent system:
\begin{subequations} \label{lin.equiv}
\beqa 
\label{lin.equiv.1}
\bmat{ccc} Q & A^T & C^T \\ 
A & 0 & 0 \\
C & 0 & -Z^{-1} S 
\emat \bmat{c} \Dx \\ -\Dy \\ -\Dz \emat & = &
\bmat{c} -r_Q \\ -r_A \\ -r_C - Z^{-1} r_{z,s} \emat, \\
\label{lin.equiv.2}
\Ds & = & Z^{-1} (-r_{z,s} - S \Dy ).
\eeqa
\end{subequations}
Application of a direct factorization code for symmetric indefinite
matrices to this equivalent form is an effective strategy. The {\tt
factor} routine would perform symmetric ordering, pivoting, and
computation of the factors, while {\tt solve} would use these factors
to solve \eqnok{lin.equiv.1} and then substitute into
\eqnok{lin.equiv.2} to recover $\Ds$.

Another possible approach is to perform another step of block
elimination and obtain a further reduction to the form
\beq \label{lin.equiv.compact}
\bmat{cc} Q + C^T Z S^{-1} C & A^T \\ A & 0 \emat
\bmat{c} \Dx \\ -\Dy \emat =
\bmat{c} -r_Q - C^T S^{-1} (Z r_C + r_{z,s}) \\ -r_A \emat.
\eeq
%
The main operation in {\tt factor} would then be to apply a symmetric
indefinite factorization procedure to the coefficient matrix in this
system, while {\tt solve} would perform triangular substitutions to
solve \eqnok{lin.equiv.compact} and then substitute to recover $\Dz$
and $\Ds$ in succession. This variant is less appealing than the
approach based on \eqnok{lin.equiv.1}, however, since the latter
approach allows the factorization routine to compute its own pivot
sequence, while in \eqnok{lin.equiv.compact} we have partially imposed
a pivot ordering on the system by performing the additional step of
block elimination. However, if the problem \eqnok{qp} contained no
equality constraints (that is, $A$ and $b$ null), the approach
\eqnok{lin.equiv.compact} might be useful, as it would allow a symmetric
{\em positive definite} factorization routine to be applied to the
matrix $Q + C^T Z S^{-1} C$.

Alternative implementations of the {\tt factor} and {\tt solve}
classes for \eqnok{qp} could apply iterative methods such as
QMR~\cite{Fre93,FreN91} or GMRES~\cite{Wal89} (see also
Kelley~\cite{ctk:roots}) to the system \eqnok{lin.equiv.1}. Under this
scenario, the role of the {\tt factor} routine is limited to choosing
a preconditioner. Since some elements of the diagonal matrix $Z^{-1}
S$ approach zero while others approach $\infty$, a diagonal scaling
that avoids the resulting ill conditioning should form part of the
preconditioning strategy.

\subsubsection{Specializing \texttt{ProblemFormulation}}
\label{specializing-problem-formulation}

Once a user has created new subclasses of \texttt{Data},
\texttt{Variables}, \texttt{Residuals}, and \texttt{LinearSystem}
appropriate to the new QP formulation, he or she must create a
subclass of \texttt{ProblemFormulation} to assemble a compatible set
objects to be used by a QP solver.  Assembly might seem to be a simple
task not requiring the use of an additional assembly class, but in
practice the process of creating a compatible set of objects can
become quite involved, as we now discuss.

Consider our example QP formulation~\eqnok{qp}. Even in this simple
case, one must create all vectors and matrices so that they have
compatible sizes and so that they are able to copy or wrap the given
problem data. The more abstract and flexible a problem formulation is,
the more options tend to be present when the objects are created. If
we wish to create a subclass of \texttt{Variables} for our new QP
formulation in which the code is independent of whether the solver is
executed on a uniprocessor platform or on a multiprocessor platform
with distributed data, we must make some other arrangements to ensure
that when the instance of \texttt{Variables} is created, the storage
for the variables is allocated and distributed in the appropriate way.
A traditional approach for managing this kind of complexity is to
isolate the code for creating a compatible set of components in a
separate subroutine. In OOQP, we use the same principle, isolating the
code for managing the complexity in the methods of a subclass of
\texttt{ProblemFormulation}.

The abstract \texttt{ProblemFormulation} class has the following
prototype.
\begin{verbatim}
class ProblemFormulation {
public:
  // makeData will often take parameters.
  //  virtual Data          * makeData()      = 0;
  virtual Residuals     * makeResiduals( Data * prob_in ) = 0;
  virtual LinearSystem  * makeLinsys( Data * prob_in )    = 0;
  virtual Variables     * makeVariables( Data * prob_in ) = 0;
  virtual ~ProblemFormulation() {};
};
\end{verbatim}
The \texttt{makeVariables} method is responsible for creating an
instance of a subclass of \texttt{Variables} that is appropriate for
this problem structure and for the computational platform. The other
methods have similar purposes for instances of the other subclasses in
the problem formulation layer.  An advantage to encapsulating the
creation code in a {\tt ProblemFormulation} class is that it is not
necessary to specify how many copies of each object need be created.
This additional flexibility is useful because different QP algorithms
need different numbers of instances of variable and residual classes.

Normally, an instance of \texttt{ProblemFormulation} will be given any
parameters that it needs to build a compatible set of objects when it
is created. Take, for example, the class \texttt{QpExampleDense},
which is a subclass of \texttt{ProblemFormulation} used to create
objects for solving QPs of the form \eqnok{qp} using dense linear
algebra. A partial prototype for the \texttt{QpExampleDense} class is
as follows.
\begin{verbatim}
class QpExampleDense : public ProblemFormulation {
 protected:
  int mNx, mMy, mMz;
 public: 
  QpExampleDense( int nx, int my, int mz );
};
\end{verbatim}
When a \texttt{QpExampleDense} is created by code of the form
\begin{verbatim}
QpExampleDense * qp = new QpExampleDense( nx, my, mz );
\end{verbatim}
it records the problem dimensions $n$, $m_A$, and $m_C$, allowing it
subsequently to create objects of the right size.

Note that the \texttt{ProblemFormulation} class does not contain
the declaration of an abstract \texttt{makeData} method. One normally
needs additional information to create \texttt{Data} objects, namely,
the problem data itself. A \texttt{makeData} method with no parameters
is normally useless; on the other hand,  no one set of
parameters would be useful for all formulations. Therefore,
there is no appropriate abstract definition of \texttt{makeData}.

%%% Local Variables: 
%%% mode: latex
%%% TeX-master: "ooqp-userguide"
%%% End: 

\newcommand\mV{\texttt{mV}}
\newcommand\IotrAddRef{\texttt{IotrAddRef}}
\newcommand\IotrRelease{\texttt{IotrRelease}}
\newcommand\SimpleVector{\texttt{SimpleVector}}
\newcommand\SmartPointer{\texttt{SmartPointer}}
\newcommand\DenseGenMatrix{\texttt{DenseGenMatrix}}
\newcommand\DenseSymMatrix{\texttt{DenseSymMatrix}}
\newcommand\SparseGenMatrix{\texttt{SparseGenMatrix}}
\newcommand\SparseSymMatrix{\texttt{SparseSymMatrix}}

\section{Using Linear Algebra Objects}
\label{sec.using-linear-algebra}

% \subsection{Intent of this Tutorial}

This section takes the form of a tutorial on elements of OOQP's linear
algebra layer. It is intended for those who wish to use these linear
algebra objects and operations concretely to define a new problem
formulation.  We have found these objects useful in implementing
solvers for the problem formulations supplied with the OOQP
distribution, and we believe they will also be useful to users who wish
to implement solvers for their own special QP formulations. Users are
not, however, compelled to use the OOQP linear algebra layer in
implementing their own problem formulation layer; they may write their
own code to store the data objects and to perform the linear algebra
operations that are required by the interior-point algorithm.

The QP formulations and interior-point algorithms supplied with the
OOQP distribution are written in terms of linear algebra operations in
abstract classes, such as \texttt{OoqpVector}, \texttt{GenMatrix}, and
\texttt{SymMatrix}. When we speak of using linear algebra objects
``concretely,'' we mean accessing the data contained in these objects
directly, in a manner that depends explicitly on how the data is
stored. A code development process using concrete objects and
operations is as follows. The user starts by creating objects that are
instances of specific concrete subclasses of the abstract linear
algebra classes, and manipulates these objects accordingly. Then, the
user migrates to an abstract interface by systematically replacing the
data-structure-dependent code in the problem formulation with
mathematical operations from the abstract base classes. Finally, the
user changes the type declarations of the variables from the
concrete classes to abstract base classes such as
\texttt{OoqpVector}, causing the compiler to disallow any remaining
data-structure-dependent code. This development process of migrating
from a working concrete QP formulation to an abstract QP formulation
may be simpler than trying to use the abstract interface on the first
pass. The material in this section will be helpful for users that
follow this path.

We start in Section~\ref{sec.ref.counting} by describing the reference
counting scheme used to manage memory in OOQP. In
Section~\ref{sec.using-simplevector}, we describe \SimpleVector, a
class that can be used in place of arrays of double-precision numbers.
Section~\ref{sec.dense.matrix} describes classes for storing and
manipulating dense matrices, while Section~\ref{sec.using-sparse}
discusses classes for sparse matrices.

\subsection{Reference Counting}
\label{sec.ref.counting}

% All objects in the linear algebra layer are reference counted.
Reference counting is a powerful technique for managing memory that
helps prevent objects from being deleted accidentally or more than
once. The technique is not limited to C++ code and, despite its name,
is unrelated to the C++ concept of reference variables.  Rather, the
term means that we maintain a count of all ``owning references'' to an
object and delete the object when this count becomes zero.  An owning
reference is a typically a pointer to an object that is a data member
of an instance of another class. Consider, for instance, the following
class.
\begin{verbatim}
class MyVariables : Variables {
    SimpleVector * mV;
public:
    SimpleVector&  v();
    SimpleVector * getV();
    void copyV( SimpleVector& w );
    MyVariables();
    MyVariables( SimpleVector * v );
    ~MyVariables();
};
\end{verbatim}
Instances of \texttt{MyVariables} would hold an owning reference to a
\SimpleVector\ in the variable \mV. In the reference counting scheme,
the destructor for this class would be as follows.
\begin{verbatim}
MyVariables::~MyVariables()
{
    IotrRelease( &mV );
};
\end{verbatim}
Rather than deleting \mV, the destructor signals that it is no longer
holding a reference to the object, so the reference count associated
with this object is decremented. In correct code, every object has at
least one owning reference. When the number of owning references has
decreased to zero through calls to \IotrRelease, the reference
counting scheme 
% automatically
deletes the object.

Usually, objects are created in the constructors of other objects and
are released when the creating object no longer needs them, typically
in the destructor.  For instance, the constructor
\begin{verbatim}
MyVariables::MyVariables()
{
    mV = new SimpleVector(5);
}
\end{verbatim}
creates a new \SimpleVector\ object, and the corresponding destructor
will release the owning reference to this object when the {\tt
MyVariables} object is finished with it.

Another common scenario is that a pointer to an object may be passed
as a parameter to a method or constructor for another object, which
may then wish to establish its own owning reference for the parameter
object. This scenario arises in the following constructor.
\begin{verbatim}
MyVariables::MyVariables(SimpleVector * v_in )
{
    mV = v_in;
    IotrAddRef( &mV );
}
\end{verbatim}
The call to \IotrAddRef\ informs the reference counting scheme that a
new owning reference to the \SimpleVector\ has been established, so
the counter associated with this object is incremented. If
\IotrAddRef\ had not been called, the reference counting scheme would
assume that the object had declined to establish a new owing
reference.

When objects are passed into methods as C++--style reference
variables, rather than via pointers, owning references must not be
established. For instance, the method
\begin{verbatim}
void MyVariables::copy( SimpleVector& w )
{
...
}
\end{verbatim}
may not establish a new owing reference for its parameter
\texttt{w}. A similar convention exists for the return values of
functions. A return value that is a C++--style reference variable
needs no special attention
\begin{verbatim}
SimpleVector& MyVariables::v()
{
    return *mV;
}
\end{verbatim}
but if the return value is a pointer, then a new owning reference is
{\em always} established, and so the reference count must be
incremented via a call to \IotrAddRef:
\begin{verbatim}
SimpleVector * MyVariables::getV()
{
    IotrAddRef( &mV );
    return mV;
}
\end{verbatim}

A typical program makes few calls to \IotrAddRef\ and \IotrRelease.
For the most part, one may simply call the \IotrRelease\ function
instead of the C++ operator \verb-delete-.

Finally, we mention that OOQP contains a {\tt SmartPointer} class that
handles calls to \IotrAddRef\ and \IotrRelease\ automatically. This
class has proven useful to the OOQP developers and is present in the
OOQP distribution for others who wish to use it. We will not,
however, describe it further in this document.

\subsection{Using \SimpleVector} \label{sec.using-simplevector}

\SimpleVector\ is a class whose instances may be used in place of
arrays of double precision numbers. It is a subclass of OOQP's
abstract base vector class, \texttt{OoqpVector}, and all abstract
operations of an \texttt{OoqpVector} are implemented in \SimpleVector.
However, there is one important additional feature: The operator
\verb-[]- has been defined for \SimpleVector, which allows indexing to
be used to access individual elements in the \SimpleVector\ object.
For example, the following piece of code involving \SimpleVector\
objects {\tt a}, {\tt b}, and {\tt c} is legal, provided that these
vectors have compatible lengths.
\begin{verbatim}
void add( SimpleVector& a, SimpleVector& b, SimpleVector& c )
{
     for( int i = 0; i < a->length(); i++ ) {
         c[i] = a[i] + b[i];
     }
}
\end{verbatim}

The elements of a \SimpleVector\ may be passed to a legacy C routine
in the manner demonstrated in the following code fragment, which calls
the C routine \texttt{norm} on the elements of \texttt{a}.
\begin{verbatim}
extern "C"
double norm( double a[], int len );
double mynorm( SimpleVector& a )
{
     return norm( a->elements(), a->length() );
}
\end{verbatim}
(Indeed, in most cases, we could use the calling sequence
\begin{verbatim}
norm( &a[0], a->length() );
\end{verbatim}
but this call will fail for vectors of length zero.) 

\SimpleVector\ objects may be created via calls to a constructor of the
following form:
\begin{verbatim}
    // Create a vector of length 5
    SimpleVector * a = new SimpleVector( 5 );
\end{verbatim}
When interfacing with non-OOQP code, however, it may be preferable to
invoke an alternative constructor that uses an existing array of
doubles to store the elements of the new \SimpleVector\ instance. Use
of this constructor is demonstrated by the following code fragment.
\begin{verbatim}
    double * v = new double[5];
    SimpleVector * b = new SimpleVector( v, 5 );
\end{verbatim}
The array \texttt{v} will be used as the storage location for the
elements of \texttt{b} and will not be deleted when \texttt{b} is
deleted.

We recommend that users always use operator \texttt{new} to create new
instances of \SimpleVector. Creating \SimpleVector\ on the stack is
not supported and may cause unforeseen problems. In other words, users
should not create variables of type \SimpleVector, but rather should
create pointers and references to instances of \SimpleVector, as in
the examples above.

\subsection{Using \DenseGenMatrix\ and \DenseSymMatrix}
\label{sec.dense.matrix}

\DenseGenMatrix\ is a class that represents matrices stored as a dense
array in row-major order. \DenseSymMatrix\ also stores matrix elements
in a dense array but represents symmetric (rather than general)
matrices. Row and columns indices for the matrices start at zero,
following C and C++ conventions.

The indexing operator \verb-[]- is defined appropriately for both
\DenseGenMatrix\ and \DenseSymMatrix. The following code fragment, for
example, is legal.
\begin{verbatim}
int myFunc( DenseGenMatrix& M )
{
  for( int i = 0; i < M.rows(); i++ ) {
    for( int j = 0; j < M.columns(); j++ ) M[i][j] = i * 10 + j;
  }
}
\end{verbatim}

\DenseSymMatrix\ stores its elements in the lower triangle of the
matrix; the result of accessing the upper triangle is undefined.  An
example of code to fill a \DenseSymMatrix\ is the following.
\begin{verbatim}
int mySymFunc( DenseSymMatrix& M )
{
  for( int i = 0; i < M.size(); i++ ) {
    for( int j = 0; j <= i; j++ ) M[i][j] = i * 10 + j;
  }
}
\end{verbatim}

The elements of a dense matrix may be passed to legacy C code by
invoking the method {\tt elements}, which returns a pointer to the
full matrix laid out in row major order. An example is as follows.
\begin{verbatim}
void myFactor( DenseGenMatrix& M )
{
    factor( M.elements(), M.rows(), M.columns() );
}
\end{verbatim}


Both \DenseGenMatrix\ or \DenseSymMatrix\ provide the method
\texttt{mult}, which performs matrix-vector multiplication. For
instance, if \texttt{M} is an instance of either class, the function
\begin{verbatim}
void func(double beta,  SimpleVector& y, 
          double alpha, SimpleVector& x)
{
    M.mult( beta, y, alpha, x )
}
\end{verbatim}
perform the computation $y \gets \beta y + \alpha M x$.  Similarly,
\texttt{transMult} computes $y \gets \beta y + \alpha M^{T} x$.

These classes contain no member functions to factor the matrices.
Users may either program their own factorization on the elements of
the matrix or use one of the linear solvers from the OOQP
distribution. For a \DenseSymMatrix\ an appropriate linear solver is
\texttt{DeSymIndefSolver}. We demonstrate the use of this solver in
the following sample code, which solves a linear system with a
coefficient matrix \texttt{M} (an instance of \DenseSymMatrix) and
right-hand side \texttt{x} (an instance of \SimpleVector). The result
is returned in the \SimpleVector\ object \texttt{y}.
\begin{verbatim}
void mySolve( SimpleVector& y, DenseSymMatrix * M, 
              SimpleVector& x )
{
     DeSymIndefSolver * solver = new DeSymIndefSolver( M );

     solver->matrixChanged();
     y.copyFrom( x );
     solver->solve( y );

     IotrRelease( &solver );
}
\end{verbatim}
The \texttt{matrixChanged} method performs an in-place factorization
on the values of \texttt{M}, overwriting the original values of this
matrix with the values of its factors. The \texttt{solve} method uses
the factors to compute the solution to the system. 

If it is known that \texttt{M} is positive definite, the solver
\texttt{DeSymPSDSolver} should be used in place of
\texttt{DeSymIndefSolver}. OOQP does not supply linear solvers for
instances of \DenseGenMatrix.

A \DenseGenMatrix\ may be created by using the operator \texttt{new}. The
following code will create a \DenseGenMatrix\ with five rows and three
columns.
\begin{verbatim}
DenseGenMatrix * pgM = new DenseGenMatrix( 5, 3 );
\end{verbatim}
Instances of \DenseSymMatrix\ are necessarily square, so only one
argument is needed for the constructor. The following code creates a
\DenseSymMatrix\ with five rows and columns.
\begin{verbatim}
DenseSymMatrix * psM = new DenseSymMatrix( 5 )
\end{verbatim}
As in the \SimpleVector class, other constructors can be invoked
to use an existing array of doubles as storage space for the
new \DenseGenMatrix\ or \DenseSymMatrix\ instances, as demonstrated in
the following code fragment.
\begin{verbatim}
double * gen = new double[5 * 3]
double * sym = new double[5 * 5];
DenseGenMatrix * pgM = new DenseGenMatrix( gen, 5, 3 );
DenseSymMatrix * psM = new DenseSymMatrix( sym, 5 )
\end{verbatim}
The arrays \texttt{gen} and \texttt{sym} will not be deleted when the
matrices \texttt{pgM} and \texttt{psM} are created or
freed.

\subsection{Using \SparseGenMatrix\ and \SparseSymMatrix}
\label{sec.using-sparse}

In many practical instances, the matrices used to formulate the QP are
large and sparse. General sparse matrices and sparse symmetric
matrices are represented by \SparseGenMatrix\ and \SparseSymMatrix,
respectively. Unlike their dense counterparts, \SparseGenMatrix\ and
\SparseSymMatrix\ cannot be used as drop-in replacements for an array
of doubles because they do not define the indexing operator \verb-[]-.

The data elements of these matrices are stored in a standard
compressed format known as {\em Harwell-Boeing} format. In the
\SparseGenMatrix\ class, the elements are stored in row-major order,
and, as in the dense case, the row and column indices start at zero.
Harwell-Boeing format encodes the matrix in three arrays---two arrays
of integers and one array of doubles. For an {\tt m} $\times$ {\tt n}
general matrix containing {\tt len} nonzero elements, these arrays are
represented by three data structures within the sparse matrix classes,
as follows.
\begin{verbatim}
int    krowM[m+1];
int    jcolM[len];
double     M[len];
\end{verbatim}
For each index ${\tt i}=0,1,\dots,m-1$, the nonzero elements from row
\texttt{i} are stored in locations \texttt{krowM[i]} through
\texttt{krowM[i+1]-1} of the vector \texttt{M}. (Recall that we index
the rows and columns by $0,1,\dots,m-1$ and $0,1,\dots,n-1$,
respectively.) The column index of each nonzero is stored in the
corresponding location of \texttt{jcolM}. In other words, for any
\texttt{k} between \texttt{krowM[i]} and \texttt{krowM[i+1]-1}
(inclusive) the \texttt{(i,jcolM[k])} element of the matrix is stored
in \texttt{M[k]}.

For a symmetric matrix, an instance of \SparseSymMatrix\ stores only
the nonzero elements in the lower triangle of the matrix. Otherwise
the format is identical to that described above for general matrices.

% The integer \texttt{krowM[i]} is the index in \texttt{jcolM}
% and \texttt{M} at which the column number and value of the first
% element of the $i^{\rm th}$ row of the matrix is stored. The integer
% \texttt{krowM[m]} is the total number of non-zeros in the matrix.

Perhaps the simplest way to understand the format is to study the
following code sample, which prints out the elements of the matrix in
row-major order.
\begin{verbatim}
for( int i = 0; i < m; i++ ) {
   for( int k = krowM[i]; k < krowM[i+1]; k++ ) {
       cout << "Row: "   << i    << "column: " << jcolM[k]
            << "value: " << M[k] << endl;
   }
}
\end{verbatim}

As for the dense classes, the \SparseGenMatrix\ and \SparseSymMatrix\
classes provide \texttt{mult} and \texttt{transMult} methods, which
perform matrix-vector multiplications. 

No methods within the sparse matrix classes perform factorizations of
the matrices. Classes with this functionality are supplied elsewhere
in the OOQP distribution, however. The default sparse direct linear
equation solver in the OOQP distribution is the code MA27 from the
Harwell Sparse Library, wrapped in a way that makes it callable from
C++ code. The following code fragment solves a system of linear
equations involving a sparse symmetric indefinite matrix. On input,
\texttt{M} contains the coefficient matrix while \texttt{x} contains
the right-hand side. Neither \texttt{M} nor \texttt{x} is changed in
the call, but \texttt{y} is replaced by the solution of the linear
system.
\begin{verbatim}
void mySparseSolve( SimpleVector& y, SparseSymMatrix * M,
                    SimpleVector& x )
{
     Ma27Solver * solver = new Ma27Solver( M );

     solver->matrixChanged();
     y.copyFrom( x );
     solver->solve( y );

     IotrRelease( &solver );
}
\end{verbatim}

Users are also free to supply their own sparse solvers. If the
solver accepts Harwell-Boeing format, the three arrays that encode the
matrix can be passed individually as arguments, as can the elements of
the right-hand side and the solution. If \texttt{M} is a
\SparseGenMatrix\ or \SparseSymMatrix\ object and \texttt{x} and
\texttt{y} are \SimpleVector\ objects, then the interface to the
user-supplied solve routine may be as follows.
\begin{verbatim}
mySparseSolver( M.krowM(), int m, M.jcolM(), M.M(), 
                x.elements(), y.elements() );
\end{verbatim}

In interior-point algorithms, one frequently must solve a
sequence of linear systems in which the matrices differ from each
other only in the diagonal elements. Consequently, we supply the
methods \texttt{fromGetDiagonal} and \texttt{atPutDiagonal}, whose
function is to transfer the diagonal elements between an instance of a
sparse matrix class and an instance of the \SimpleVector\ class.  For
example, the following code copies diagonal elements $(4,4)$ through
$(8,8)$ inclusive from the \SparseSymMatrix\ object \texttt{M} into
the \SimpleVector object \texttt{d}.
\begin{verbatim}
SimpleVector * getMe( SparseSymMatrix& M )
{
    SimpleVector * d = new SimpleVector(5);
    M.fromGetDiagonal( 4, *d );
    return d;
}
\end{verbatim}
To copy the elements from \texttt{d} into the diagonals of \texttt{M},
one would use a call of the form
\begin{verbatim}
M.atPutDiagonal( 4, *d );
\end{verbatim}
which overwrites diagonal elements $(4,4)$ through $(3+r,3+r)$ with
the elements of \texttt{d}, where $r$  is the number of elements
in \texttt{d}.

Instances of \SparseGenMatrix\ or \SparseSymMatrix\ can be created by
first filling the three arrays that encode the matrix in
Harwell-Boeing format and then calling a constructor. For general
sparse matrices, this call has the following form:
\begin{verbatim}
SparseGenMatrix * sgm 
   = new SparseGenMatrix( m, n, len, krowM, jcolM, M );
\end{verbatim}
where \texttt{m} and \texttt{n} are the number of rows and columns,
respectively, \texttt{len} is the number of nonzero elements, and
\texttt{krowM}, \texttt{jcolM}, and \texttt{M} are the three arrays
discussed above.  For sparse matrices, the corresponding call is
\begin{verbatim}
SparseSymMatrix * ssm 
    = new SparseSymMatrix( m, len, krowM, jcolM, M );
\end{verbatim}
We emphasize that the arrays \texttt{krowM}, \texttt{jcolM}, and
\texttt{M} are not copied but rather are used directly. They are not
deleted when the sparse matrix instances \texttt{sgm} or \texttt{ssm}
are freed.

Alternative constructors can be used when a description of the matrix
is available in {\em sparse triple} format. In this simple format, the
matrix is encoded in two integer arrays and one double array, all of
which have length equal to the number of nonzeros in the matrix. (In
the case of a symmetric matrix, only the lower triangle of the matrix
is stored.) By defining \texttt{nnz} to be the number of stored
nonzeros, and defining the three arrays as follows,
\begin{verbatim}
int irow[nnz];
int jcol[nnz];
double A[nnz];
\end{verbatim}
we have for any \texttt{k} in the range \texttt{0,...,nnz-1} that the
element at row \texttt{irow[k]} and column \texttt{jcol[k]} has value
\texttt{A[k]}. The elements in this format can be sorted into
row-major order by calling another routine from the OOQP distribution,
\texttt{doubleLexSort}, in the following way.
\begin{verbatim}
doubleLexSort( irow, nnz, jcol, A );
\end{verbatim}

Given the matrix in this form, with the arrays sorted into row-major
form, we can build an instance of \SparseGenMatrix\ or
\SparseSymMatrix\ by first calling a constructor with the matrix
dimensions and the number of non-zeros as arguments, as follows.
\begin{verbatim}
SparseGenMatrix * sgm = new SparseGenMatrix( m, n, nnz );
SparseSymMatrix * ssm = new SparseSymMatrix( m, nnz );
\end{verbatim}
We can then call the method \texttt{putSparseTriple}, available in
both classes, to place the information in \texttt{irow},
\texttt{jcol}, and \texttt{A} into \texttt{sgm} or \texttt{ssm}. This
call has the following form.
\begin{verbatim}
sgm.putSparseTriple( irow, nnz, jcol, A, info ); 
\end{verbatim}
The output parameter \texttt{info} will be set to zero if \texttt{sgm}
is large enough to hold the elements in \texttt{irow}, \texttt{jcol},
and \texttt{A}. Otherwise it will be set to one.

%%% Local Variables: 
%%% mode: latex
%%% TeX-master: "ooqp-userguide"
%%% End: 

\section{Specializing Linear Algebra Objects}
\label{sec.specializing-linalg}

The solver supplied in the OOQP distribution for the formulation
\eqnok{qpgen} with sparse data uses the MA27~\cite{duff82ma27} sparse
indefinite linear equation solver from the Harwell Subroutine Library
to solve the systems of linear equations that arise at each
interior-point iteration. Some users may wish to replace MA27 with a
different sparse solver. (Indeed, we implemented a number of
different solvers during the development of OOQP.) Users also may want
to make other modifications to the linear algebra layer supplied with
the distribution. For example, it may be desirable to alter the
representations of matrix and vectors that are implemented in OOQP's
linear algebra layer, by creating new subclasses of
\texttt{OoqpVector}, \texttt{SymMatrix}, \texttt{GenMatrix}, and
\texttt{DoubleStorage}. One motivation for doing so might
be to embed OOQP in an applications code that defines its own
specialized matrix and vector storage schemes.

In Section~\ref{sec:new.linear.solver}, we describe the process of
replacing MA27 by a new linear solver.
Section~\ref{sec:special.matvec} discusses the subclassing of objects
in OOQP's linear algebra layer that may be carried out by users who
wish to specialize the representations of matrices and vectors.

\subsection{Using a Different Linear Equation Solver}
\label{sec:new.linear.solver}

The MA27 solver for symmetric indefinite systems of linear equations
is an efficient, freely available solver from the Harwell Subroutine
Library that is widely used to solve the linear systems that arise in
the interior-point algorithm applied to sparse QPs of the form
\eqnok{qpgen}. By the nature of OOQP's design, however, an advanced
user can substitute another solver without much trouble. This section
outlines the steps that must be taken to do so.  We focus on replacing
a sparse linear solver because this operation is of greater practical
import than replacing a dense solver and because there are a
greater variety of sparse factorization codes than of dense codes.

\subsubsection{Creating a Subclass of {\tt DoubleLinearSolver}}
\label{sec.subclass.DoubleLinearSolver}

The first step is to create a subclass of the
\texttt{DoubleLinearSolver} class. A typical subclass will have the
following prototype.
\begin{verbatim}
#include "DoubleLinearSolver.h"
#include "SparseSymMatrix.h"
#include "OoqpVector.h"

class MyLinearSolver : public DoubleLinearSolver {
  SparseSymMatrix * mStorage;
public:
  MyLinearSolver( SparseSymMatrix * storage );
  virtual void diagonalChanged( int idiag, int extent );
  virtual void matrixChanged();
  virtual void solve ( OoqpVector& vec );
  virtual ~MyLinearSolver();
};
\end{verbatim}
Each \texttt{DoubleLinearSolver} object is associated with a matrix.
Therefore, a typical constructor for a subclass
\texttt{MyLinearSolver} of \texttt{DoubleLinearSolver} would be as
follows.
\begin{verbatim}
MyLinearSolver::MyLinearSolver( SparseSymMatrix * ssm )
{
    IotrAddRef( &ssm );
    mMat = ssm; // Here mMat is a data member of MyLinearSolver.
}
\end{verbatim}
The call to \texttt{IotrAddRef} establishes an owning reference to the
matrix (see Section~\ref{sec.ref.counting}). It must be balanced by a
call to \texttt{IotrRelease} in the destructor, as follows.
\begin{verbatim}
MyLinearSolver::~MyLinearSolver()
{
    IotrRelease( &mMat );
}
\end{verbatim}

When the linear solver is first created, the matrix with which it is
associated will not typically contain any data of interest to the
linear solver. Once the contents of the matrix have been loaded, the
interior-point algorithm may call the \texttt{matrixChanged} method,
which triggers a factorization of the matrix.  Subsequently, the
algorithm performs one or more calls to the \texttt{solve} method,
each of which uses the matrix factors produced in
\texttt{matrixChanged} to solve the linear system for a single
right-hand side.

Calls to \texttt{matrixChanged} typically occur once at each
interior-point iteration. It is assumed that the sparsity structure of
the matrix does not change between calls to \texttt{matrixChanged};
only the data values will be altered. This assumption, which holds for
all popular interior-point algorithms, allows subclasses of
\texttt{DoubleLinearSolver} to cache information about the sparsity
structure of the matrix and its factors and to reuse this information
throughout the interior-point algorithm.

The \texttt{diagonalChanged} method supports those rare solvers that
take a different action if only the diagonal elements of the matrix
are changed (while off-diagonals are left untouched). Most solvers
cannot do anything interesting in this case; a typical implementation
of \texttt{diagonalChanged} simply calls \texttt{matrixChanged}, as
follows.
\begin{verbatim}
void MyLinearSolver::diagonalChanged( int idiag, int extent )
{
   this->matrixChanged();
}
\end{verbatim}

The implementation of \texttt{matrixChanged} and \texttt{solve}
depends strongly on the sparse linear system solver in use, as well as
on the data format used to store the sparse matrices.
Section~\ref{sec.using-sparse} describes the data format used by our
sparse matrix classes. The convention in OOQP is that sparse linear
solvers must not act destructively on the matrix data.  In some
instances, this restriction requires a copy of part of the matrix data
to be made before factorization begins. Typically, however, this
restriction is not too onerous because the fill-in that occurs during
a typical factorization would make it necessary to allocate additional
storage in any case.

The opposite convention is in place for subclasses of
\texttt{DoubleLinearSolver} that operate on dense matrices. These
invariably perform the factorization in place, overwriting the matrix
data. While having two different conventions is far from ideal, we
felt it unwise to enforce unnecessary copying of matrices in the dense
case for the sake of conformity.

\subsubsection{Creating a Subclass of {\tt ProblemFormulation}}
\label{sec.subclass.ProblemFormulation}

Having defined and implemented a new subclass of {\tt
  DoubleLinearSolver}, the user must now arrange so that the new
solver, rather than the default linear solver, is created and used by
the quadratic programming algorithm. 


In Section~\ref{sec:linsysclass} we described how subclasses of
\texttt{LinearSystem} are used to solve the linear systems arising in
interior point algorithms. We give specific examples of how an
instance of \texttt{LinearSystem} designed to handle our example QP
formulation~\eqnok{qp} assembles a matrix and right-hand side of a
system to be passed to a general-purpose linear solver, which would
normally be an instance of a subclass of
\texttt{DoubleLinearSolver}. In this manner, we have separated the
problem-specific reductions and transformations, which are the
responsibility of instances of \texttt{LinearSystems}, from the
solution of matrix equations, which are the responsibility of
instances of \texttt{DoubleLinearSolver}.

On the other hand, the nature and properties of the
\texttt{DoubleLinearSolver} will affect the efficiency and feasibility
of problem-specific reductions and transformations.  Moreover, when
the \texttt{LinearSystem} assembles the matrix equations to be solved,
it must assemble the matrix in a format acceptable to the linear
solver. To ensure that a compatible set of objects is created, the
\texttt{DoubleLinearSolver}, the matrix it operates on, and
\texttt{LinearSystem} are created in the same routine.

As we discussed in Section~\ref{specializing-problem-formulation}, OOQP
contains classes---specifically, subclasses of
\texttt{ProblemFormulation}---that exist for the express purpose of
creating a compatible set of objects for implementing solvers for QPs
with a given formulation. The \texttt{makeLinsys} methods of these
classes is, naturally, the place in which appropriate instances of
subclasses of \texttt{LinearSystem} are created. As we discussed in
the earlier section, code for creating a compatible collection of
objects can become quite involved, so it is natural to collect this
code in one place. OOQP's approach is to place this code in the
methods of subclasses of \texttt{ProblemFormulation}.

To use a new \texttt{DoubleLinearSolver} with an existing problem
formulation, one must create a new subclass of
\texttt{ProblemFormulation}. Since the code needed to implement a
subclass of \texttt{ProblemFormulation} depends strongly on the
specific data structures of the problem formulation, it is difficult
to give general instructions on how to write such code. However, we
describe below the appropriate procedure for users who wish to work
with a sparse variant of the {\tt QpGen} formulation \eqnok{qpgen},
changing only the {\tt DoubleLinearSolver} object and retaining the
data structures and other aspects of the formulation that are used in
the default (MA27-based) solver supplied with the OOQP
distribution. To accommodate such users, we have created a subclass of
\texttt{ProblemFormulation} called \texttt{QpGenSparseSeq}, which
holds the code common to all formulations of \texttt{QpGen} that uses
sparse sequential linear algebra.  Users can create a subclass of
\texttt{QpGenSparseSeq} in the following way.
\begin{verbatim}
class QpGenSparseMySolver : public QpGenSparseSeq {
public:
  QpGenSparseMySolver( int nx, int my, int mz,
               int nnzQ, int nnzA, int nnzC );
  LinearSystem * makeLinsys( Data * prob_in );
};
\end{verbatim}
The constructor may be implemented by simply passing its arguments
through to the parent constructor.
\begin{verbatim}
QpGenSparseMySolver::QpGenSparseMySolver( int nx, int my, int mz,
                      int nnzQ, int nnzA, int nnzC ) :
  QpGenSparseSeq( nx, my, mz, nnzQ, nnzA, nnzC )
{
}
\end{verbatim}
The implementation of the \texttt{makeLinsys} method is too
solver-specific to be handled by generic code, but the following code
fragment, which is based on the file {\tt
src/QpGen/QpGenSparseMa27.C}, may give a useful guide.
\begin{verbatim}
LinearSystem * QpGenSparseMySolver::makeLinsys( Data * prob_in )
{
  QpGenData * prob = (QpGenData *) prob_in;
  int n = nx + my + mz;

  // Include diagonal elements in the matrix, even if they are
  // zero. Enforce by inserting a diagonal of zeros.
  SparseSymMatrix *  Mat = 
     new SparseSymMatrix( n, n + nnzQ + nnzA + nnzC );

  SimpleVector * v = new SimpleVector(n);  
  v->setToZero();
  Mat->setToDiagonal(*v);                  
  IotrRelease( &v );

  prob->putQIntoAt( *Mat, 0, 0 );
  prob->putAIntoAt( *Mat, nx, 0);
  prob->putCIntoAt( *Mat, nx + my, 0 );
  // The lower triangle is now [ Q   * ]
  //                           [ A   C ]
  
  MyLinearSolver  * solver = new MyLinearSolver( Mat );
  QpGenSparseLinsys  * sys 
      = new QpGenSparseLinsys( this, prob, 
                               la, Mat, solver );

  IotrRelease( &Mat );

  return sys;
}
\end{verbatim}
We emphasize that users who wish to alter the MA27-based
implementation of the solver for the sparse variant of \eqnok{qpgen}
only by substituting another solver with similar capabilities to MA27
will be able to use these examples directly, by inserting the names
they have chosen for their solver into these code fragments.

% One must then arrange so that the new solver, rather than the default
% linear solver, is created and used by the quadratic programming
% algorithm. The linear solver is used by a class in the problem
% formulation layer, a subclass of \texttt{LinearSystem}, to solve the
% systems needed by the QP algorithm.  Subclasses of
% \texttt{LinearSystem} assemble the matrix data of the linear system to
% be solved, and use a \texttt{DoubleLinearSolver} to solve equations
% using this data. The operations needed to assemble the matrix are
% specific to the problem formulation, which is why
% \texttt{LinearSystem} is part of the problem formulation layer (see
% Section~\ref{sec:linsysclass}.) In OOQP, instances of subclasses of
% \texttt{ProblemFormulation} are responsible for creating a compatible
% set of objects to define a problem formulation.  Thus, the proper and
% well-defined way to create a \texttt{LinearSystem} that uses the new
% linear equation solver is to create a new subclass of
% \texttt{ProblemFormulation}.
% 
% The specific method of \texttt{ProblemFormulation} that must be
% overridden is \texttt{makeLinsys}.  However, as a rule, most of the
% code needed to implement a subclass of \texttt{ProblemFormulation} is
% closely tied to the specific data structures of the problem
% formulation. This makes it difficult to give general instructions on
% how to override \texttt{makeLinsys}. Therefore, in this section, we
% will focus specifically on the \texttt{QpGen} formulation.

\subsection{Specializing the Representation of Vectors and Matrices}
\label{sec:special.matvec}

Although the OOQP linear algebra layer provides a comprehensive set of
linear algebra classes, as described in
Section~\ref{sec.using-linear-algebra}, some users may wish to
use a different set of data structures to represent vectors and
matrices. This could happen, for instance, when the user needs to
embed OOQP in a larger program with its own data structures already
defined.  The design of OOQP is flexible enough to accommodate
user-defined linear algebra classes. In this section, we outline how
such classes can be written and incorporated into the code.

The vector and matrix classes need to provide methods that, for the
most part, represent simple linear algebra operations, such as inner
products and saxpy operations.  The names are often
self-explanatory; those that are specific to the needs of the
interior-point algorithm are described in the class documentation
accompanying the OOQP distribution. We note, however, that efficient
implementation of these operations can require a significant degree of
expertise, especially when the data structures are complex. We
recommend that users search for an existing implementation that is
compatible with their data storage needs before attempting to
implement the methods themselves. As a rule, it is easier to create
OOQP vectors and matrix classes that wrap existing libraries than to
write efficient code from scratch.

To specialize the representation of vectors and matrices, one must
create subclasses of the following abstract classes:
\begin{description}
  \item[OoqpVector:] Represents mathematical vectors.
  \item[GenMatrix:] Represents nonsymmetric and possibly
    nonsquare matrices as mathematical operators.
  \item[SymMatrix:] Represents symmetric matrices as mathematical operators.
  \item[DoubleStorage:] Contains the concrete code for managing the
    data structures that hold the matrix data.
  \item[DoubleLinearSolver:] Solves linear systems with a specific type
    of matrix as its coefficient.
  \item[LinearAlgebraPackage:] Creates instances of vectors and matrices.
\end{description}
We have outlined how to create a new subclass of
\texttt{DoubleLinearSolver} in the preceding section. The remainder
of this section will focus on the other new subclasses. We will not
describe the methods of these classes in detail, because the majority
of them are familiar mathematical operations. We refer the reader to
the class documentation accompanying the OOQP distribution for a
description of these methods.

The code in the  problem formulation layer is implemented b using the
abstract linear algebra classes described above. Objects in the
problem formulation layer can be created by using instances of
user-defined subclasses to represent linear algebra objects. We have
discussed in the preceding section and in
Section~\ref{specializing-problem-formulation} the use of the
\texttt{ProblemFormulation} class in creating a compatible set of
objects in the problem formulation layer. Users who wish
to specialize the representation of vectors and matrices will also
need to create at least one new subclass of
\texttt{ProblemFormulation}.

The header file \texttt{src/Vector/OoqpVector.h} defines the abstract
vector class. The header files defining the other abstract classes may
be found in the subdirectory \texttt{src/Abstract}. As a rule, the
files needed to define a particular implementation of the linear
algebra layer are given their own subdirectory. Some existing
implementations are located in the following directories.
\begin{verbatim}
    src/DenseLinearAlgebra/
    src/SparseLinearAlgebra/
    src/PetscLinearAlgebra/
\end{verbatim}
Users may wish to refer to these implementations as sample code.
Because \texttt{DenseLinearAlgebra} and \texttt{SparseLinearAlgebra}
share the same vector implementation, \texttt{SimpleVector}, this code
is located in its own directory, named \texttt{src/Vector}.
Several linear solvers have also been given their own subdirectories
below the directory \texttt{src/LinearSolvers}.

OOQP does not attempt to force matrices and vectors that are
represented in significantly different ways to work together properly.
For instance, the distribution contains no method that multiplies a
matrix stored across several processors by a vector whose data is
stored on a tape drive attached to a single processor. Nor do we
perform any compile-time checks that only compatible linear algebra
objects are used together in a particular implementation.  Such checks
would require heavy use of the C++ template facility, and we were wary
of using templates because of the portability issues and other costs
that might arise. Rather, we endeavored to design our problem
formulation classes in a way that makes it difficult to mix
representations of linear algebra objects accidentally.  (We suggest
that users who are modifying the matrix and vector representations
follow this design.)  Commonly, we downcast at the start of a method.
For example, the following code fragment downcasts from the abstract
{\tt OoqpVector} class to the {\tt MyVector} class, which the {\tt
  mult} method in {\tt MySymMatrix} is intended to use.
\begin{verbatim}
void MySymMatrix::mult ( double beta,  OoqpVector& y_in,
                double alpha, OoqpVector& x_in )
{
  MyVector & y = (MyVector &) y_in;
  MyVector & x = (MyVector &) x_in;
}
\end{verbatim}

Subclasses of \texttt{DoubleStorage} are responsible for the physical
storage of matrix data on a computer. The physical data structure
might be as simple as a dense two-dimensional array. In a
distributed-computing setting, it could be much more complex. 
% Because
% the methods that must be implemented by matrix and vector subclasses
% are strongly representation-dependent, i
Instances of \texttt{DoubleStorage} are rarely used in an abstract
setting. The code will know precisely what type of
\texttt{DoubleStorage} is being used and what concrete data structures
are being used to implement it. Thus, many of the methods of a
subclass of \texttt{DoubleStorage} will be data-structure specific.

By contrast, each subclass of \texttt{DoubleStorage} will be
associated with subclasses of \texttt{GenMatrix} and
\texttt{SymMatrix} that are used primarily in an abstract,
data-structure-independent fashion. Subclasses of \texttt{GenMatrix}
and \texttt{SymMatrix} generally implement their methods by calling
the structure-specific methods of a subclass of
\texttt{DoubleStorage}. By using this design in OOQP, we were able to
separate abstract mathematical manipulations of matrices and vectors
from details of their representation. Accordingly, in creating their
subclasses, users should feel free to implement any
structure-dependent methods they need in their implementation of the
\texttt{DoubleStorage} subclass, whereas their implementations of the
\texttt{GenMatrix} and \texttt{SymMatrix} subclasses should adhere
more closely to the abstract interface.

We emphasize the following points for users who wish to create
subclasses from the matrix classes: Matrices in OOQP are represented
in row-major form, and row and column indices start at zero. Adherence
to these conventions will make it easier to refer to existing
implementations in designing new versions of the linear algebra layer.
Symmetric matrices in OOQP store their elements in the lower triangle
of whatever data structure is being used. For some linear algebra
implementations, it might be desirable to symmetrize the structure,
explicitly storing all elements of the matrix, despite the redundancy
this entails. If this approach is chosen, one should be careful to
treat the matrix as if only the lower triangle were significant, as
subtle bugs may arise otherwise.

Subclasses of \texttt{OoqpVector} represent mathematical vectors and
should adhere closely to the abstract vector interface. The methods of
\texttt{OoqpVector} typically operate on the entire vector. Access
to individual elements of the vector should be avoided.

Users who implement their own representation of vectors and matrices
will also need to specialize the \texttt{LinearAlgebraPackage}
class. This class has the following interface (see {\tt
src/Abstract/LinearAlgebraPackage.h}).
\begin{verbatim}
class LinearAlgebraPackage {
protected:
  LinearAlgebraPackage() {};
  virtual ~LinearAlgebraPackage() {};
public:
  virtual SymMatrix * newSymMatrix( int size, int nnz ) = 0;
  virtual GenMatrix * newGenMatrix( int m, int n, int nnz ) = 0;
  virtual OoqpVector * newVector( int n ) = 0;
  // Access the type name for debugging purposes.
  virtual void whatami( char type[32] ) = 0;
};
\end{verbatim}
Instances of \texttt{LinearAlgebraPackage} do nothing more than create
vectors and matrices on request. 
% It might appear strange to have a 
% class with no other responsibilities than this. 
Our reason for including this class in the OOQP design is to provide a
mechanism by which abstract code can create new vectors and matrices
that are compatible with existing objects. The code cannot call the
operator \texttt{new} on a type name and still remain abstract. Use of
\texttt{LinearAlgebraPackage}, on the other hand, allows users to
create new vectors and matrices, without referring to specific vector
and matrix types, by invoking the \texttt{newVector},
\texttt{newSymMatrix}, and \texttt{newGenMatrix} methods of an
instance of \texttt{LinearAlgebraPackage}.

Instances of \texttt{LinearAlgebraPackage} are never deleted. Because
these instances are small, the memory overhead is normally
insignificant. However, it is customary to arrange so that each
subclass of \texttt{LinearAlgebraPackage} has at most one instance, as
in the following code fragment.
\begin{verbatim}
class MyLinearAlgebraPackage : public LinearAlgebraPackage {
protected:
  DenseLinearAlgebraPackage() {};
  virtual ~DenseLinearAlgebraPackage() {};
public:
  static MyLinearAlgebraPackage * soleInstance();
  // ...
}

MyLinearAlgebraPackage * MyLinearAlgebraPackage::soleInstance()
{
  static 
  MyLinearAlgebraPackage * la = new MyLinearAlgebraPackage;

  return la;
}
\end{verbatim}
The use of such a scheme is optional. 


%%% Local Variables: 
%%% mode: latex
%%% TeX-master: "ooqp-userguide"
%%% End: 



\addcontentsline{toc}{section}{References}
\bibliographystyle{plain}
\bibliography{refs}
\newpage
\appendix
\section{COPYRIGHT}%
\index{COPYRIGHT}

\subsection*{COPYRIGHT NOTIFICATION}%
\index{COPYRIGHT NOTIFICATION}

The following is a notice of limited availability of this software and
disclaimer which must be included as a preface to the software, in all
source listings of the code, and in any supporting documentation.
\\ \\
COPYRIGHT 2001 UNIVERSITY OF CHICAGO
\\ \\
The copyright holder hereby grants you royalty-free rights to use,
reproduce, prepare derivative works, and to redistribute this software
to others, provided that any changes are clearly documented.  This
software was authored by:
\begin{quote}
\begin{tabular}{l}
    E. MICHAEL GERTZ      gertz@mcs.anl.gov \\
    Mathematics and Computer Science Division \\
    Argonne National Laboratory \\
    9700 S. Cass Avenue \\
    Argonne, IL 60439-4844 \\ \\

    STEPHEN J. WRIGHT     swright@cs.wisc.edu \\
    Computer Sciences Department \\
    University of Wisconsin \\
    1210 West Dayton Street \\
    Madison, WI 53706   FAX: (608)262-9777 
\end{tabular}
\end{quote}
Any questions or comments may be directed to one of the authors.

ARGONNE NATIONAL LABORATORY (ANL), WITH FACILITIES IN THE STATES OF
ILLINOIS AND IDAHO, IS OWNED BY THE UNITED STATES GOVERNMENT, AND
OPERATED BY THE UNIVERSITY OF CHICAGO UNDER PROVISION OF A CONTRACT
WITH THE DEPARTMENT OF ENERGY.

\subsection*{GOVERNMENT LICENSE}%
\index{GOVERNMENT LICENSE}

Portions of this material resulted from work developed under a
U.S. Goverment contract and are subject to the following license: the
Government is granted for itself and others acting in its behalf a
paid-up, nonexclusive, irrevocable worldwide license in this computer
software to reproduce, prepare derivative works, distribute copies to
the public, and perform publicly and display publicly.

\subsection*{DISCLAIMER}%
\index{DISCLAIMER}

NEITHER THE UNITED STATES GOVERNMENT NOR ANY AGENCY THEREOF, NOR THE
UNIVERSITY OF CHICAGO, NOR ANY OF THEIR EMPLOYEES, MAKES ANY WARRANTY,
EXPRESS OR IMPLIED, OR ASSUMES ANY LEGAL LIABILITY OR RESPONSIBILITY
FOR THE ACCURACY, COMPLETENESS, OR USEFULNESS OF ANY INFORMATION,
APPARATUS, PRODUCT, OR PROCESS DISCLOSED, OR REPRESENTS THAT ITS USE
WOULD NOT INFRINGE PRIVATELY OWNED RIGHTS.

%%% Local Variables: 
%%% mode: latex
%%% TeX-master: "ooqp-userguide"
%%% End: 


\end{document}

%%% Local Variables: 
%%% mode: latex
%%% TeX-master: t
%%% End: 
